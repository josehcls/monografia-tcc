\documentclass[]{politex}
% ========== Opções ==========
% pnumromarab - Numeração de páginas usando algarismos romanos na parte pré-textual e arábicos na parte textual
% abnttoc - Forçar paginação no sumário conforme ABNT (inclui "p." na frente das páginas)
% normalnum - Numeração contínua de figuras e tabelas 
%	(caso contrário, a numeração é reiniciada a cada capítulo)
% draftprint - Ajusta as margens para impressão de rascunhos
%	(reduz a margem interna)
% twosideprint - Ajusta as margens para impressão frente e verso
% capsec - Forçar letras maiúsculas no título das seções
% espacosimples - Documento usando espaçamento simples
% espacoduplo - Documento usando espaçamento duplo
%	(o padrão é usar espaçamento 1.5)
% times - Tenta usar a fonte Times New Roman para o corpo do texto
% noindentfirst - Não indenta o primeiro parágrafo dos capítulos/seções


% ========== Packages ==========
\usepackage[utf8]{inputenc}
\usepackage{amsmath,amsthm,amsfonts,amssymb}
\usepackage{graphicx,cite,enumerate}
\usepackage{mhchem}
\usepackage{listings}
\usepackage{float}
\restylefloat{table}
\usepackage[dvipsnames]{xcolor}
\usepackage[hyphens]{url}
\usepackage[breaklinks=true, colorlinks=true, linkcolor=black, urlcolor=blue, citecolor=black]{hyperref}

% ========== Commands ==========
\newcommand*{\captionsource}[2]{%
  \caption[{#1}]{%
    #1%
    \\\hspace{\linewidth}%
    Fonte: #2%
  }%
}

% ========== Language options ==========
\usepackage[brazil]{babel}
%\usepackage[english]{babel}


% ========== ABNT (requer ABNTeX 2) ==========
%	http://www.ctan.org/tex-archive/macros/latex/contrib/abntex2
%\usepackage[num]{abntex2cite}

% Forçar o abntex2 a usar [ ] nas referências ao invés de ( )
%\citebrackets{[}{]}


% ========== Lorem ipsum ==========
\usepackage{blindtext}



% ========== Opções do documento ==========
% Título
\titulo{Sistema de Monitoramento e Controle da Fermentação de Cervejas}

% Autor
\autor{José Henrique Camargo Leopoldo e Silva\\
       Rafael Jardim Pastor}

% Para múltiplos autores (TCC)
%\autor{Nome Sobrenome\\%
%		Nome Sobrenome\\%
%		Nome Sobrenome}

% Orientador / Coorientador
\orientador{Prof. Dr. Carlos Eduardo Cugnasca}
% \coorientador{Nome do coorientador (opcional)}

% Tipo de documento
\tcc{Eletricista}
%\dissertacao{Engenharia Elétrica}
%\teseDOC{Engenharia Elétrica}
%\teseLD
%\memorialLD

% Departamento e área de concentração
\departamento{Nome do departamento}
\areaConcentracao{Engenharia Elétrica}

% Local
\local{São Paulo}

% Ano
\data{2020}




\begin{document}
% ========== Capa e folhas de rosto ==========
\capa
\falsafolhaderosto
\folhaderosto


% ========== Folha de assinaturas (opcional) ==========
%\begin{folhadeaprovacao}
%	\assinatura{Prof.\ X}
%	\assinatura{Prof.\ Y}
%	\assinatura{Prof.\ Z}
%\end{folhadeaprovacao}


% ========== Ficha catalográfica ==========
% Fazer solicitação no site:
%	http://www.poli.usp.br/en/bibliotecas/servicos/catalogacao-na-publicacao.html


% ========== Dedicatória (opcional) ==========
% \dedicatoria{Dedicatória}


% ========== Agradecimentos ==========
% \begin{agradecimentos}

% Thanks...

% \end{agradecimentos}


% ========== Epígrafe (opcional) ==========
% \epigrafe{%
% 	\emph{``Epígrafe''}
% 	\begin{flushright}
% 		-{}- Autor
% 	\end{flushright}
% }


% ========== Resumo ==========
\begin{resumo}
Na produção de cervejas, a fermentação alcoólica é um complexo bioquímico cuja função primária é converter os açúcares extraídos dos grãos maltados em álcool etílico. Esse processo tem grande impacto no sabor, aroma, aparência e textura do produto, sendo o processo mais importante durante a produção da bebida. 
As principais variáveis desse processo são: a temperatura, que influencia no metabolismo das leveduras, a densidade, que indica a evolução da fermentação, e o pH, que é um indicador adicional de qualidade. 
O objetivo desse projeto é criar um protótipo de um dispositivo que realize o monitoramento dessas variáveis e controle a temperatura, garantindo resultados mais precisos e reprodutíveis. O sistema é voltado para cervejarias de pequeno e médio porte que desejem monitorar e controlar o processo durante os testes de receitas em pequena escala de produção, e também para produtores \textit{hobbyistas} que almejem maior controle sobre o processo e resultados mais consistentes. 
O sistema idealizado é composto por dois subsistemas complementares: (i) um físico: agregando o controlador, hardware com sensores e atuadores e software embarcado; e (ii) um digital: que capta, processa e disponibiliza todas as informações adquiridas para o usuário por meio de uma plataforma online. A execução do projeto foi guiada por iterações curtas e prototipagem, focando na implementação gradativa de funcionalidades a cada iteração até a construção total do protótipo.
 % Deste modo, para a aquisição de conhecimento técnico e de negócio necessários para a execução do projeto foi realizado um levantamento e estudo de material bibliográfico adequado, entrevistas com mestres-cervejeiros para o qual o produto se destinaria e análise de soluções já existentes no mercado, dentro e fora do país.
% // TODO: Revisar Resumo
%
\\[3\baselineskip]
%
\textbf{Palavras-Chave} -- Produção de Bebidas, Fermentação Alcoólica, Cerveja, Controle de Processos.
\end{resumo}


% // TODO: Escrever Abstract
% ========== Abstract ==========
% \begin{abstract}
% Abstract...
% %
% \\[3\baselineskip]
% %
% \textbf{Keywords} -- Word, Word, Word, Word, Word.
% \end{abstract}


% ========== Listas (opcional) ==========
\listadefiguras
\listadetabelas

% ========== Listas definidas pelo usuário (opcional) ==========
% // TODO: Separar Lista de Símbolos em um arquivo e colocar (símbolos químicos, constantes, etc.)
% \begin{pretextualsection}{Lista de símbolos}

% Lista de símbolos...

% \end{pretextualsection}

% ========== Sumário ==========
\sumario
% // TODO: Sumário sem subsubsection / 4o nível


% ========== Elementos textuais ==========

% \part{Introdução}
	
% \chapter{Capítulo com epígrafe}
% \capepigrafe[0.5\textwidth]{``Frase espirituosa de um autor famoso''}{Autor famoso}

% \blindtext

% \begin{citacaoLonga}
% 	\blindtext
% \end{citacaoLonga}

% \blindtext



% \blinddocument

\chapter{Introdução}

\section{Objetivo}

O objetivo geral deste estudo é desenvolver um protótipo de um sistema que realize o monitoramento e controle do processo de fermentação de cervejas.
O objetivo específico é possibilitar às cervejarias de pequeno e médio porte o desenvolvimento da capacidade de testes de novas receitas de cervejas, 
a fim de garantir a reprodutibilidade e qualidade das mesmas por meio do controle do processo de fermentação.

% // TODO: Escrever Motivação
\section{Motivação}
\textcolor{red}{\{Comentários sobre a motivação de escolha do tema.\}}

\section{Organização do Trabalho}

% // TODO: Escrever Organização do Trabalho
\textcolor{red}{\{Breve descrição de como o trabalho está organizado, como Contextualização, Metodologia empregada, Projeto, Implantação e Considerações finais\}}



\chapter{Contexto}

Esse projeto tem natureza multidisciplinar, buscando harmonizar a bioquímica do
processo de produção de cervejas com sistemas de controle e automação e
geração e análise de dados estudados na Engenharia Elétrica. Em sua primeira
parte, foi estudado o funcionamento das leveduras e como o ambiente age sobre
seu processo metabólico, enquanto na outra parte, foram estudados sensores,
atuadores e arquitetura de Internet das Coisas para captação, processamento e
disponibilização de informações referente a ação dos microrganismos.

\section{História da Cerveja}

% // TODO: Colocar fontes bibliográficas

% A cerveja pode ser considerada a bebida alcoólica mais antiga do mundo, e atualmente tem uma importância social e econômica muito grande. Durante anos, a sua produção ganhou diversas mudanças, melhorias e adaptações até chegar nas variedades de forma e sabor conhecidas atualmente. 


Segundo \citeonline{Kunze}, a cerveja provavelmente teve origem na revolução agrícola, na qual os humanos começaram a abandonar o nomadismo e se estabeleceram em comunidades que cultivam diversos grãos. Com o armazenamento de cereais, é provável que essa origem tenha sido acidental, com uma fermentação espontânea da cevada. Os mais antigos registros dessa bebida podem ser encontrados na antiga região mesopotâmia, atual Irã, e são datados em 2800 a.C. A sua importância em comunidades antigas do oriente médio era tanta, que em 1760 a.C., foi criada a primeira lei que regulamenta a produção e venda de cerveja. A lei conhecida como Estela de Hamurabi regulamentou a comercialização, fabricação e consumo, estabelecendo uma ração diária de cerveja para os habitantes da região. 


Como o consumo da cerveja era mais seguro do que a água, visto que o processo de fervura ajuda a purificar a bebida, seu consumo em algumas sociedades era visto como uma necessidade básica diária, continua \citeonline{Kunze}. Esse fato ajudou a aumentar a popularidade da bebida na Europa durante a idade média, principalmente nas comunidades Germânicas. Uma das instituições mais importantes para o desenvolvimento dos processos de produção eram os mosteiros dessa região e época. Os monges eram responsáveis pela fabricação da bebida e como eram os únicos que reproduziam os manuscritos, puderam conservar e aperfeiçoar a sua produção, sendo muito influentes até hoje. Eles foram responsáveis por incluir diversas ervas na fabricação, sendo a mais importante delas o lúpulo, utilizado para trazer o amargor da bebida. Uma das leis mais conhecidas e importantes da indústria cervejeira é a da pureza Alemã. Devido a diversas mudanças aplicadas pelos fabricantes e a percepção de uma queda de qualidade, essa lei foi criada no século XIV na região da Bavária (Sul da Alemanha) buscando uma padronização da bebida. A lei instituiu que a cerveja deveria ser fabricada apenas com os seguintes ingredientes: água, malte de cevada e lúpulo. Atualmente, apesar da legalização do uso de qualquer ingrediente na região, muitos cervejeiros ainda seguem essa restrição e ela é um sinônimo de qualidade. 


Ainda segundo \citeonline{Kunze}, uma das mais importantes inovações na fabricação foi a Pasteurização, que permite a preservação do gosto por mais tempo. O processo consiste, basicamente, no aquecimento da bebida a uma determinada temperatura, por determinado tempo, e depois a bebida é resfriada de forma a eliminar os microorganismos ali presentes. O processo foi nomeado pelo seu criador o francês Louis Pasteur que atendendo a solicitação de alguns dos vinicultores e cervejeiros da região que lhe pediram para descobrir como os vinhos e a cervejas azedaram. Durante sua investigação, através do uso de microscópio, ele pôde constatar que a levedura ocasionava este processo e assim criou esse processo de purificação. A partir dele, a indústria cervejeira conseguiu chegar em um novo nível e crescer em escala e alcance, sendo essencial para grandes fábricas atualmente. 


% Atualmente, a indústria de cerveja movimenta milhões de dólares anualmente e uma das divisões que mais cresce são as microcervejarias. No Brasil começaram esses pequenos produtores começaram a surgir na década de 90. Em 2012 as cervejas especiais representavam 8\% do mercado nacional da bebida em 2012 e encerraram 2014 com uma participação de 11\%, segundo o Sindicato Nacional da Indústria da Cerveja, que aponta a existência de 300 microcervejarias no País. A projeção é de que essa cota suba para 20\% em 2020.


\section{Fabricação de Cervejas}

\citeonline{Lewis} apresenta a produção de cerveja como uma atividade que deve equilibrar 
séculos de tradição e arte, desenvolvida por gerações de cervejeiros, e a abordagem científica 
e avanços tecnológicos, de forma que estes possam proporcionar maior entendimento, controle e melhorias 
sobre o processo mas sem abandonar suar raízes históricas e descaracterizá-lo. 
Justamente, as etapas principais do processo ainda seguem a produção tradicional, apesar
da evolução de métodos e técnicas adicionais.

Nesta seção é descrito o processo mais comum de produção de cervejas, considerando os principais
ingredientes: água, malte, lúpulo e levedura. Tradicionalmente, são utilizados grãos maltados de cevada,
mas atualmente podem ser utilizados outros grãos, maltados ou não, como por exemplo: trigo, milho, arroz e aveia.
A partir da descrição de \citeonline{Kunze}, são listadas as principais etapas de produção:


\begin{enumerate}
    \item Maltagem dos grãos: processo de germinação parcial e controlada dos grãos, com a finalidade
de produzir enzimas como a amilase. A germinação é interrompida por um processo de secagem quando atinge o estágio desejado.
A partir desse momento, os grãos passam a ser denominados malte;
    \item Moagem do malte: quebra do malte em pequenos fragmentos para expor as enzimas e componentes internos. É desejável
que parte da casca seja mantida intacta para auxiliar na filtragem após a mostura;
    \item Mostura: o malte moído é misturado em água, e aquecida em temperaturas
que estimulem a ação das enzimas obtidas na maltagem. As enzimas
realizam a quebra de moléculas insolúveis de amido em moléculas menores
de açúcares, que são dissolvidas, formando uma solução denominada mosto;
    \item \textit{Lautering}: o mosto é separado do restante do malte que não foi dissolvido. As
cascas dos grãos auxiliam essa etapa formando um filtro natural, mas também podem ser utilizados filtros;
    \item Fervura: o mosto é fervido, consequentemente: a ação enzimática é
interrompida e a solução é esterilizada. Nessa etapa, lúpulos são adicionados em diferentes
momentos da fervura, fornecendo extratos que conferem amargor e aroma à cerveja;
    \item Fermentação: após o resfriamento do mosto, leveduras são adicionadas e a solução é aerada. 
Os açúcares são consumidos pelo metabolismo das leveduras, gerando etanol e dióxido de carbono;
    \item Maturação: após o consumo dos açúcares, as leveduras passam a reabsorver parte dos
subprodutos gerados durante a fermentação, melhorando a qualidade geral da bebida. Ao final do processo, as leveduras floculam e decantam, podendo ser extraídas e reaproveitadas;
    \item Envase: transferência para o recipiente final. Nessa etapa, também é
realizada a carbonatação da bebida, geralmente por injeção de gás carbônico
ou por meio de uma segunda fermentação, aproveitando-se leveduras ainda presentes na solução.
Industrialmente, é comum a filtragem pré-envase, e a pasteurização pós-
envase.
\end{enumerate}

\subsection{Fermentação e Maturação}

\citeonline{Kunze} posiciona a fermentação como o processo mais importante da produção de cervejas, sendo as etapas
antecedentes responsáveis para garantir as condições necessárias para que as leveduras possam converter os açúcares,
extraídos do malte e quebrados em moléculas menores pelas enzimas, em álcool e gás carbônico. As leveduras consomem 
os açúcares para gerar energia e se reproduzirem, e nesse processo também geram centenas de subprodutos como ésteres, álcoois pesados e compostos sulfúricos que mesmo em pequena quantidade são essenciais para as características organolépticas da cerveja, como
pontuam \citeonline{YeastWhite}.

Leveduras são organismos unicelulares membros do reino fungi. Na produção de cervejas são utilizadas principalmente 
duas espécies: a \textit{Saccharomyces cerevisie} (em cervejas do tipo \textit{Ale}) e a \textit{Saccharomyces pastorianus} (em cervejas do tipo \textit{Lager}). Ao serem inoculadas ao mosto, as leveduras rapidamente absorvem o oxigênio dissolvido para revitalizar sua membrana celular e iniciar a absorção dos açúcares e nutrientes do mosto, descrevem \citeonline{YeastWhite}.
A partir do açúcar absorvido, a levedura pode convertê-lo em energia por meio da respiração aeróbica que ocorre na presença de oxigênio, ou pela respiração anaeróbica que ocorre na ausência de oxigênio ou em solução com alta concentração de glicose (pelo chamado efeito de Crabtree). 
Além da produção energia no formato de moléculas de adenosina trifosfato (ATP), a respiração anaeróbica também produz etanol e gás carbônico, por isso também é denominada de fermentação alcoólica; a equação \ref{eq:fermentacao} ilustra essa reação. A molécula de ATP provém energia para a síntese de proteínas e replicação de DNA, necessários para a multiplicação celular.

\begin{equation}
    Glicose + 2 ADP + 2 Fosfato \longrightarrow 2 Etanol + 2 CO_2 + 2 ATP
    \label{eq:fermentacao}
\end{equation}

A etapa de fermentação e maturação de uma cerveja leva em média entre 7 e 14 dias, desde a inoculação das leveduras até a cerveja estar pronta para envase. Vale notar que podem ser aplicados processos extras na maturação que prolonguem esse tempo.
Para uma fermentação básica, \citeonline{YeastWhite} classificam o processo em 3 fases, que ocorrem com certo grau de sobreposição:
\begin{enumerate}
    \item Fase de retardamento, durante as primeiras quatro a 15 horas após a adição
da levedura no mosto, caracterizada pela climatização das células ao
ambiente e preparação para a próxima fase;
    \item Fase de crescimento exponencial, que dura entre quatro horas e quatro dias,
quando ocorre o consumo dos açúcares e replicação logarítmica das células
de leveduras;
    \item Fase estacionária, em que o crescimento diminui e alguns compostos, como o diacetil, são
absorvidos pelas leveduras, maturando o produto durante três a dez dias.
\end{enumerate}

A variação nas durações das etapas é decorrente tanto da quantidade e saúde das leveduras inoculadas, quanto da temperatura 
durante a fermentação. De acordo com \citeonline{YeastWhite}, tradicionalmente, cervejas do tipo \textit{Ale} são fermentadas a 20°C e cervejas do tipo 
\textit{Lager} a 10°C. Temperaturas inferiores tornam o processo mais lento, e em casos extremos culminar na parada do processo, impedindo a fermentação completa ou que os subprodutos como o diacetil seja absorvido pelas leveduras. temperaturas superiores aceleram a atividade celular, tornando o processo mais rápido, mas também podendo fazer com que as leveduras se multipliquem demais e gerem uma quantidade de subprodutos alta que pode impactar negativamente na qualidade da bebida, gerando sabores indesejados denominados \textit{off-flavors}. 
Justamente por essa razão, o controle de temperatura durante a fermentação é essencial para a obtenção de um produto de 
qualidade com resultados consistentes, principalmente nas primeiras 72 horas do processo, que representam o pico de multiplicação celular, geração de calor decorrente da atividade celular e produção de sub-produtos.

Após a temperatura, \citeonline{YeastWhite} colocam a densidade relativa e pH do mosto como segunda e terceira
variáveis mais importantes para se monitorar durante a fermentação.
A densidade relativa que indica o grau de evolução do processo. Como parte do gás carbônico produzido 
na equação \ref{eq:fermentacao} é dissolvido e capturado pela atmosfera, a densidade do mosto diminui durante a 
fermentação à medida em que os açúcares vão sendo consumidos. A partir disso, é possível: identificar as etapas da 
fermentação pelo gráfico de evolução da densidade relativa ao longo do tempo; estimar a graduação alcoólica a partir dos 
valores inicial e final da densidade relativa; e estipular o grau de atenuação, ou quanto dos açúcares foram consumidos 
pelas leveduras.
Enquanto o pH, que sofre uma leve queda durante o processo, pode ser 
utilizado principalmente como uma amostra das condições iniciais da fermentação, com intuito de se obter reprodutividade do 
processo, e também pode servir como uma variável de diagnóstico de problemas que podem ocorrer como não adaptação das 
leveduras ao mosto e contaminação.

Com caráter ilustrativo, é incluído gráfico com o perfil dessas e outras variáveis durante a fermentação de uma cerveja 
do tipo \textit{Lager} (Figura \ref{fig:variaveis_fermentacao}), retirado de \citeonline{FermentationMunroe}. Na figura, \textit{specific gravity} se refere à
densidade relativa do mosto, e \textit{cells in suspension} à quantidade de células de levedura em suspensão, vale notar que 
ao final do processo, as leveduras floculam e decatam, por isso a queda.

\begin{figure}[H]
    \centering
    \includegraphics[scale=0.40]{figuras/contexto/variaveis_fermentacao.PNG}
    \captionsource{Gráfico de evolução de variáveis ao longo da fermentação.}{\citeonline{FermentationMunroe}}
    \label{fig:variaveis_fermentacao}
\end{figure}

\section{Análise de Soluções Existentes}


Em produções de grande escala, \citeonline{YeastWhite} afirmam que a maior parte das cervejarias utiliza grandes fermentadores cônicos, que possuem um sistema de resfriamento em seu revestimento, com fluido circulante, para controlar a temperatura durante a fermentação. Para pequenas produções, há uma grande variedade de métodos e equipamentos disponíveis para realizar esse controle, desde sistemas sofisticados que replicam os fermentadores industriais, até métodos manuais.


Empresas como a Grainfather, Ss Brewtech e Birchmann oferecem sistemas de fermentação de primeira linha, com controle de temperatura utilizando a circulação de líquido refrigerante. São modelos relativamente caros, que se assemelham ao equipamento profissional de grande escala. A figura \ref{fig:ssbrewtech} esquematiza um desses produtos, constituído de um compressor para resfriamento do líquido que circula no fermentador por uma serpentina e uma resistência para aquecimento acoplada no fermentador.

\begin{figure}[H]
    \centering
    \includegraphics[scale=0.25]{figuras/contexto/glycol_ftss2_1000x.jpg}
    \captionsource{Esquema do equipamento de controle da fermentação FTSs² da Ss Brewtech.}{https://www.ssbrewtech.com/}
    \label{fig:ssbrewtech}
\end{figure}


Na outra extremidade do espectro de soluções, produtores \textit{hobbyistas} que estejam preocupados com o controle de temperatura têm desenvolvido algumas técnicas e equipamentos simples para controle. Um exemplo de técnica comum, é submergir parcialmente o fermentador em um reservatório com água e controlar a temperatura pela adição de gelo nesse reservatório, é um método de baixo custo, mas que exige acompanhamento e esforço constante para se obter um resultado mediano. 


Buscando o controle automatizado, um projeto comum na comunidade de cervejeiros caseiros, é a utilização de geladeiras ou freezers, adaptando-se seu termostato para definição de uma temperatura alvo específica. Esse último método requer um investimento maior na aquisição do refrigerador, mas atinge consegue atingir um bom resultado.
Outra abordagem encontrada é a montagem de um sistema utilizando bomba e serpentina para circulação de um líquido refrigerante. Uma alternativa de menor custo e mais simples em relação ao condensador utilizado na solução já citada, é resfriar o líquido de saída da serpentina em um recipiente com isolamento térmico contendo gelo.


Entre as soluções profissionais e as soluções \textit{hobbyistas} surge uma série de produtos com a finalidade de entregar uma solução pronta e com um custo menor em relação a um sistema completo. Esses produtos tendem a ser comercializados como acessórios que podem ser acoplados a fermentadores de diferentes tamanhos e modelos, sendo mais adaptáveis às necessidades do usuário. As empresas citadas que oferecem sistemas completos também possuem esses \textit{kits} para montagem, além de diversas outras empresas focadas em dispositivos mais simples.


Uma solução bastante interessante é o produto Immersion Pro comercializado pela Brew Jacket (figura \ref{fig:brewjacket}). Ela consiste em controlador que utiliza uma pastilha de Peltier para controle da temperatura (seja resfriamento ou aquecimento) do mosto em fermentação, através de uma haste para troca de calor que fica submersa. É um produto versátil o suficiente para ser instalado nos principais modelos de fermentador e apresenta a capacidade de manter a temperatura com diferença de até 20°C da temperatura ambiente desde que aplicado um isolamento térmico ao fermentador.

\begin{figure}[H]
    \centering
    \includegraphics[scale=0.45]{figuras/contexto/brewjacket.jpg}
    \captionsource{Produto Immersion Pro desenvolvido pela Brew Jacket.}{https://brewjacket.com/}
    \label{fig:brewjacket}
\end{figure}

Além das soluções discutidas para controle da temperatura, há uma linha de produtos especializados no monitoramento da temperatura e densidade durante a fermentação e disponibilização online das informações. Dentre as soluções disponíveis, destacam-se pela abordagem na medição da densidade, os produtos Plaato, Beer Bug e Tilt Hydrometer.

A Plaato (figura \ref{fig:plaato}) conseguiu desenvolver uma técnica singular de inferir a densidade do mosto pelo volume de \(CO_2\) que é expelido durante a fermentação. O produto na forma de um \textit{airlock} é conectado ao fermentador vedado e envia os dados de densidade e temperatura coletados via conexão \textit{Wi-Fi}, que podem ser acompanhados em computadores ou dispositivos móveis. Também com disponibilização dos dados de temperatura e densidade online, o Tilt Hydrometer (figura \ref{fig:tilt}) é um pequeno dispositivo cilíndrico que fica imerso no fermentador e infere a densidade pelo ângulo de inclinação de flutuação, enquanto que o Beer Bug (figura \ref{fig:beerbug}) é instalado como um \textit{airlock} e infere a densidade medindo a força de empuxo sofrida por uma carga submersa no mosto e conectado ao dispositivo por um fio.

Analisando as soluções existente, este projeto se posiciona com o objetivo de desenvolver um protótipo que reúna o controle de temperatura com o monitoramento online das variáveis relevantes. Além disso, que possa ser desenvolvido em um produto razoavelmente versátil para ser instalado em fermentadores comuns.

\begin{figure}[h]
    \centering
    \includegraphics[scale=0.25]{figuras/contexto/plaato.jpg}
    \captionsource{Solução Plaato.}{https://plaato.io/}
    \label{fig:plaato}
\end{figure}


\begin{figure}[h]
    \centering
    \includegraphics[scale=0.45]{figuras/contexto/tilt.png}
    \captionsource{Solução Tilt Hydrometer.}{https://tilthydrometer.com/}
    \label{fig:tilt}
\end{figure}

\begin{figure}[h]
    \centering
    \includegraphics[scale=0.35]{figuras/contexto/beerbug.png}
    \captionsource{Solução Beer Bug.}{https://straighttothepint.com/}
    \label{fig:beerbug}
\end{figure}



\chapter{Metodologia do Trabalho}

A elaboração desse projeto segue metodologia típica de projetos de controle de
processos, estudada na graduação, consistindo inicialmente de estudo sobre o
processo a ser controlado e levantamento dos requisitos funcionais e não funcionais
necessários a se atingir o objetivo deste trabalho.


A partir dos requisitos, pretende-se seguir com o desenvolvimento seguindo uma
metodologia de prototipagem guiada por iterações curtas com incremento de
funcionalidades constante. com o objetivo de acelerar a obtenção de
resultados.


Para avaliar cada iteração do protótipo, o processo de produção de cervejas,
incluindo a fermentação, foi reproduzido em pequena escala. Essa abordagem prática 
aguça o conhecimento do processo e tem grande potencial em evidenciar erros no projeto.


\chapter{Projeto}

\section{Especificação de Requisitos Técnicos}

O protótipo deve ser capaz de monitorar e controlar o processo de fermentação de cervejas, seguindo as configurações de receita definidas pelo usuário. 
Todas as informações coletadas devem ser disponibilizadas ao usuário com a finalidade de possibilitar maior entendimento e reprodutibilidade 
do processo. O projeto foi dividido em dois sistemas: um Hardware, encarregado das medições e controle, e um Software, responsável por exibir informações e estabelecer uma interface com o usuário. A partir dessas premissas, foram determinados os seguintes requisitos para cada um dos sistemas.

\subsection{Sistema Hardware}

\subsubsection{Requisitos funcionais de Hardware}

HW-F-1) O sistema deve monitorar a temperatura (entre 0 e 100 °C), com precisão de 0,5 °C e intervalo de 1 minuto.

HW-F-2) O sistema deve monitorar o pH em intervalo de 1 minuto.

HW-F-3) O sistema deve monitorar a densidade relativa (entre 1,000 e 1,150), com precisão de 0,001 em relação à água a 20°C e intervalo de 1 minuto.

HW-F-4) O sistema deve controlar a temperatura de até 50 Litros de mosto em fermentação, com desvio máximo de 0,5°C em relação ao valor definido pelo usuário e diferença máxima de 10°C em relação ao ambiente.

HW-F-5) O sistema deve seguir o perfil de controle (temperatura x tempo) definido pelo usuário no Software.

HW-F-6) Os dados monitorados devem ser enviados para o Software a cada 5 minutos por meio de rede sem fio.

\subsubsection{Requisitos não funcionais de Hardware}

HW-NF-1) O sistema deve ser acoplável a fermentadores de até 50 Litros disponíveis no mercado.

HW-NF-2) Em caso de perda de conexão com o Software, o sistema deve tentar enviar os dados ainda não enviados a cada ciclo de envio.

HW-NF-3) Caso o sistema tenha uma oscilação no fornecimento de energia, ele deve ser capaz de voltar ao funcionamento normal.

HW-NF-4) Os dados monitorados devem ser armazenados temporariamente, por no mínimo 15 dias, no Hardware.

\subsection{Sistema Software}

\subsubsection{Requisitos funcionais de Software}

SW-F-1) O sistema deve fornecer acesso ao usuário após identificação com usuário e senha

SW-F-2) O sistema deve fornecer as informações instantâneas das fermentações em progresso.

SW-F-3) O sistema deve permitir acesso às informações históricas de fermentações já realizadas.

SW-F-4) O sistema deve permitir o cadastro de receitas. Uma receita é definida por: identificação, nome, estilo e campo livre para observações. O campo livre pode evoluir para um cadastro padronizado dos ingredientes e processos realizados.

SW-F-5) O sistema deve permitir o cadastro de lotes. Um lote é definido por: identificação, receita utilizada, instante de início da fermentação, instante de fim da fermentação, variáveis personalizadas, perfil de controle e observação.

SW-F-6) O sistema deve permitir o cadastro de perfis de controle. Um perfil de controle é definido por: identificação, nome e temperatura alvo, instante (em relação ao início da fermentação).

SW-F-7) O sistema deve permitir o cadastro de variáveis personalizadas. Uma variável personalizada é definida por: identificador, lote correspondente, chave, valor e instante (em relação ao início da fermentação).

SW-F-8) O sistema deve disponibilizar, para cada lote um gráfico com a evolução de cada variável monitorada ao longo do tempo de fermentação.

SW-F-9) Os dados recebidos pelo Hardware devem ser salvos em banco de dados

SW-F-10) Em caso de perda de conexão com o Hardware, o usuário deve ser notificado por e-mail.

SW-F-11) O sistema deve permitir que o usuário realize o download de seus dados em formato de planilha.

SW-F-12) O sistema deve permitir que o usuário registre seus dispositivos.


\subsubsection{Requisitos não funcionais de Software}

SW-NF-1) As informações de cada usuário são, por padrão, particulares de cada usuário e devem seguir padrões de segurança.

SW-NF-2) As informações instantâneas devem estar disponíveis em até 1 minuto após o recebimento dos dados pelo HW.

SW-NF-3) O sistema deve ser desenvolvido na forma de Web-App, e ser responsivo a dispositivos mobile e computadores.



\section{Projeto de Controle}

\subsection{Modelagem Térmica do Sistema}

O primeiro passo para a implementação do controle de temperatura da fermentação é a modelagem térmica do processo. Nota-se que essa modelagem é bastante complexa pois envolve diversos coeficientes térmicos desconhecidos, devido a composição heterogênea do Mosto com Leveduras. 

O objetivo do sistema é controlar a temperatura do líquido fermentado (solução de mosto e leveduras). O líquido estará dentro de um fermentador, sem entrada de ar dado que a fermentação é anaeróbica e o ar interfere na qualidade do experimento. Para expulsar o $CO_2$ gerado pela fermentação e não permitir a entrada de $O_2$ o fermentador utiliza um Air-Lock, dispositivo que funciona como válvula só permitindo a direção única desse fluxo. O fermentador será embalado em uma manta térmica, com a intenção de evitar a troca térmica com o ambiente.

Na maior parte das receitas, a fermentação ocorre em temperaturas entre 14°C e 20°C, temperaturas muitas vezes inferiores à temperatura do ambiente. Sendo assim, existe a necessidade de criar um sistema que troque calor com o mosto e leveduras mantendo um gradiente de temperatura entre o fermentador e o ambiente constante conforme configuração do usuário. 

Para isso acontecer durante o processo, o calor retirado pela refrigeração deve ser o mesmo que o gerado pela convecção com o ambiente e pelo próprio processo de fermentação, que é exotérmico. 


É importante destacar que o dispositivo irá ter duas funções: a primeira será manter a temperatura do líquido fermentado; a segunda será levar o líquido até determinada temperatura. Ambas as funcionalidades envolvem a capacidade do dispositivo de retirar ou fornecer calor do sistema de forma eficiente e constante. 


\subsection{Transferência de Calor entre Ar e Fermentador}

Nessa modelagem, serão analisados os seguintes elementos: 
\begin{itemize}
    \item Mosto e Leveduras 
    \item Tanque de Fermentação 
    \item Manta Térmica 
\end{itemize}

Nessa dinâmica, considerando que o fermentador vai operar na maior parte do tempo em temperaturas inferiores às do ambiente, o calor vai fazer o seguinte caminho:
\begin{enumerate}
    \item A manta térmica recebe calor do ambiente através da convecção e radiação do ar. 
    \item A manta térmica transfere calor por condução para o fermentador. 
    \item O fermentador transfere calor por condução para o líquido fermentado. 
    \item O líquido fermentado produz calor pelo processo de fermentação (processo exotérmico).
\end{enumerate}

\begin{center}
    \(Ambiente \longrightarrow Manta\;t\acute{e}rmica \longrightarrow Fermentador \longrightarrow Mosto\;e\;leveduras\)
\end{center}



    
Algumas hipóteses foram adotadas visando simplificar o problema: 

\begin{enumerate}
    \item O gradiente de temperatura no interior do Mosto + Leveduras é desprezível; 
    \item O coeficiente de troca de calor por convecção entre o Mosto + leveduras e o tanque é elevados o bastante para que não sejam observadas diferenças de temperatura entre esses elementos; 
    \item O regime é permanente e as propriedades são constantes 
    \item A condução é unidimensional no plano X
    \item A transferência de calor por radiação é desprezível nas superfícies 
    \item Resistências de contato desprezíveis.
\end{enumerate}


As hipóteses 1, 2, 3  podem ser adotadas devido ao horizonte de tempo da fermentação, no qual é necessário manter a mesma temperatura durante dias. As hipóteses 4, 5 e 6 foram adotadas com a intenção de simplificar o problema. A figura \ref{fig:fermentador_controle} exemplifica esse problema.

\begin{figure}[h]
    \centering
    \includegraphics[scale=0.45]{figuras/projeto/controle/fermentador_controle.png}
    \captionsource{Desenho do problema de troca de calor entre o ambiente e fermentador.}{Autores}
    \label{fig:fermentador_controle}
\end{figure}



\subsection{Simulação Térmica}

A Simulação térmica do sistema possui três objetivos:

\begin{enumerate}
    \item Estimar a potência de troca de calor necessária para manter o sistema com 10°C de diferença em relação à temperatura ambiente;
    \item Estimar o tempo morto do sistema para levar o sistema do equilíbrio térmico para uma diferença de 10°C;
    \item Simular um controle PID utilizando uma fonte de calor variável, no caso do sistema real, uma pastilha de Peltier para a sintonização inicial do sistema.
\end{enumerate}

Com esses testes será possível entender se a utilização de pastilhas de Peltier será suficiente para o controle de temperatura.


Foi utilizada a biblioteca Simscape do software Matlab. Essa biblioteca contém blocos que simulam elementos térmicos e funcionam de maneira análoga a um circuito elétrico. No caso, a diferença de temperatura seria equivalente à diferença de potencial elétrico e o calor que circula no sistema seria equivalente a corrente elétrica. As figuras \ref{fig:sim_fonte_calor} a \ref{fig:sim_sensor} representam blocos utilizados.


\begin{figure}[H]
    \centering
    \includegraphics[scale=1.0]{figuras/projeto/controle/fonte_calor.png}
    \caption{Bloco de fonte de calor, que mantém a temperatura em um ponto do circuito constante.}
    \label{fig:sim_fonte_calor}
\end{figure}

\begin{figure}[H]
    \centering
    \includegraphics[scale=1.0]{figuras/projeto/controle/conveccao.png}
    \caption{Bloco de transferência de calor por convecção, seguindo a fórmula \(Q = k \cdot A \cdot (T_A - T_B) \)}
    \label{fig:sim_conveccao}
\end{figure}

\begin{figure}[H]
    \centering
    \includegraphics[scale=1.0]{figuras/projeto/controle/radiacao.png}
    \caption{Bloco de transferência de calor por radiação, seguindo a fórmula \(Q = k \cdot A \cdot (T_A^4 - T_B^4) \)}
    \label{fig:sim_radiacao}
\end{figure}

\begin{figure}[H]
    \centering
    \includegraphics[scale=1.0]{figuras/projeto/controle/conducao.png}
    \caption{Bloco de transferência de calor por condução, seguindo a fórmula \(Q = k \cdot \dfrac{A}{D} (T_A - T_B) \)}
    \label{fig:sim_conducao}
\end{figure}

\begin{figure}[H]
    \centering
    \includegraphics[scale=0.8]{figuras/projeto/controle/sensor_ideal.png}
    \caption{Bloco de sensor de calor ideal}
    \label{fig:sim_sensor}
\end{figure}

\begin{table}[H]
    \begin{center}
        \begin{tabular}{ |c|c| } 
            \hline
            Símbolo & Descrição \\
            \hline
            \(Q\) & Quantidade de Calor \\
            \hline
            \(k\) & Coeficiente de condutividade térmica do material \\
            \hline
            \(A\) & Área normal à transmissão de calor \\
            \hline
            \(D\) & Espessura do material \\
            \hline
            \(T_A, T_B\) & Temperaturas da camada A e B, respectivamente \\
            \hline
        \end{tabular}
        \caption{\label{tab:legenda_blocos} Descrição dos símbolos de trocas de calor referentes às figuras \ref{fig:sim_conveccao} a \ref{fig:sim_conducao}.}
    \end{center}
\end{table}


Com as devidas simplificações justificadas anteriormente, o sistema foi simulado com o circuito da figura \ref{fig:sim_circuito}. A seguir, cada seção de blocos do circuito é descrita em detalhes. A temperatura ambiente adotada na simulação é 25°C e a temperatura do líquido, 15°C.

\begin{figure}[H]
    \centering
    \includegraphics[scale=0.39]{figuras/projeto/controle/sim_circuito.png}
    \captionsource{Circuito de blocos utilizado para a simulação térmica.}{Autores}
    \label{fig:sim_circuito}
\end{figure}


Para a simulação, foi considerado um fermentador cilíndrico com 37 cm de altura e 26 cm de raio. Ele tem paredes de polipropileno com coeficiente térmico de \(0,25 W / m \cdot K\)  e 0,1 cm de espessura. Esse fermentador será encoberto por uma camada de isolante de espuma de poliestireno com coeficiente térmico de  \(0,03 W / m \cdot K\) e 0,2 cm de espessura. O líquido será considerado como água, visando simplificar os cálculos. 


O conjunto de blocos da figura \ref{fig:transcal_amb} representa a transferência de calor entre o ar e a camada isolante que circunda o fermentador. Foram consideradas duas formas de transferência, por convecção e radiação. A área de contato adotada foi \(2 \pi \cdot 0,37 ]cdot 0,263 m^2\), o coeficiente de radiação, \( 5,667 e^{-8} \cdot 0,7 W / m^2 \cdot K^4 \), e o coeficiente de convecção, \(25 W /m^2 \cdot K\) (ar).

\begin{figure}[H]
    \centering
    \includegraphics[scale=0.6]{figuras/projeto/controle/transcal_amb.png}
    \caption{Seção do circuito da figura \ref{fig:sim_circuito} referente à transferência de calor entre ambiente e isolamento do fermentador.}
    \label{fig:transcal_amb}
\end{figure}


O bloco da figura \ref{fig:transcal_isolante} representa a transferência de calor por condução entre as paredes do isolante térmico que circunda o fermentador. Foi considerada área de \(2\pi \cdot 0,37 \cdot 0,263 m^2\), espessura de 0,2 cm e coeficiente térmico de condução, \(0.03 W / m \cdot K\).

\begin{figure}[H]
    \centering
    \includegraphics[scale=0.8]{figuras/projeto/controle/transcal_isolante.png}
    \caption{Seção do circuito da figura \ref{fig:sim_circuito} referente à transferência de calor pelo isolamento do fermentador.}
    \label{fig:transcal_isolante}
\end{figure}


A transferência entre extremidades da parede do fermentador é simbolizada pelo bloco da figura \ref{fig:transcal_fermentador}. Para esse bloco, foi adotada área de \(2\pi \cdot 0,37 \cdot 0,261 m^2\), espessura de 0,1 cm e coeficiente térmico de condução, \(0.25 W / m \cdot K\).

\begin{figure}[H]
    \centering
    \includegraphics[scale=0.8]{figuras/projeto/controle/transcal_fermentador.png}
    \caption{Seção do circuito da figura \ref{fig:sim_circuito} referente à transferência de calor pelas paredes do fermentador.}
    \label{fig:transcal_fermentador}
\end{figure}


Finalmente, a transferência de calor entre a parede do fermentador e do mosto em fermentação está representado pelo bloco de transferência de calor da figura \ref{fig:transcal_mosto}. Para a simulação foram considerados os valores de área como \(2\pi \cdot 0,37 \cdot 0,261 m^2\) e coeficiente 1000 W / m²·K (considerando a referência da água). 

\begin{figure}[H]
    \centering
    \includegraphics[scale=0.8]{figuras/projeto/controle/transcal_mosto.png}
    \caption{Seção do circuito da figura \ref{fig:sim_circuito} referente à transferência de calor entre paredes do fermentador e mosto em fermentação.}
    \label{fig:transcal_mosto}
\end{figure}


Executando a simulação, a curva de calor resultante (figura \ref{fig:curva_calor}) indica que para manter a temperatura sistema 10°C abaixo do ambiente, é necessário retirar cerca de 57 W de calor do sistema. Esse valor é compatível com a potência máxima das pastilhas de Peltier disponíveis no mercado, que variam entre 90 e 230 W.

\begin{figure}[h]
    \centering
    \includegraphics[scale=0.30]{figuras/projeto/controle/curva_calor.jpg}
    \captionsource{Curva de calor obtida a partir da simulação do circuito da figura \ref{fig:sim_circuito}.}{Autores}
    \label{fig:curva_calor}
\end{figure}


Uma segunda simulação foi realizada para determinar o tempo necessário para levar temperatura do sistema do  equilíbrio térmico até uma novo valor. Para isso é ligada uma massa térmica, com o calor específico igual à água de \( 4184 J/K \cdot Kg \), ocupando de 15L do fermentador, e com massa de aproximadamente 15 Kg. O atuador será uma fonte de calor variável, que irá retirar do sistema uma quantidade de calor constante de 57 W. Simulando para intervalo de tempo de 20 horas, foram obtidas as curvas de temperatura (figura \ref{fig:curva_temp}) e quantidade de calor (figura \ref{fig:curva_q}).


\begin{figure}[H]
    \centering
    \includegraphics[scale=0.50]{figuras/projeto/controle/circ_atuador.png}
    \captionsource{Circuito térmico completo, com atuador e massa térmica.}{Autores}
    \label{fig:circ_atuador}
\end{figure}


Observando o sistema, conclui-se que o tempo necessário para mover o sistema do equilíbrio térmico para uma temperatura alvo é muito alto e isso pode ser prejudicial nas primeiras horas da fermentação. Dessa forma, é interessante iniciar o processo com a primeira temperatura alvo, o que já é de certa forma realizado na prática, pois a temperatura de início da fermentação deve ser atingida antes do inoculação da levedura e fechamento do fermentador.


\begin{figure}[H]
    \centering
    \includegraphics[scale=0.60]{figuras/projeto/controle/curva_temp.png}
    \captionsource{Curva de temperatura em Kelvin por tempo em dezenas de milhares de segundos, obtida da simulação do circuito térmico da figura.}{Autores}
    \label{fig:curva_temp}
\end{figure}

\begin{figure}[H]
    \centering
    \includegraphics[scale=0.60]{figuras/projeto/controle/curva_q.png}
    \captionsource{Curva de quantidade de calor em Watts por tempo em dezenas de milhares de segundos, obtida da simulação do circuito térmico da figura.}{Autores}
    \label{fig:curva_q}
\end{figure}

\subsection{Dispositivo para Troca de Calor}

Um dos maiores desafios do projeto é criar um sistema que consiga trocar calor de forma eficiente, não seja intrusivo e consiga ser simples o suficiente para ser utilizado por um hobbysta. Conforme anteriormente especificado, a ideia é utilizar placas de peltier para realizar essa troca. A maior vantagem do uso dessas placas é a possibilidade de controlar o calor associado proporcionalmente a quantidade de corrente fornecida ao módulo, através da seguinte equação \ref{eq:qp}. Possibilitando assim o uso de um circuito junto com o microcontrolador para o controle de temperatura.

\begin{equation}
    Q_P = \pi \cdot 1
    \label{eq:qp}
\end{equation}


\subsection{Controle de Temperatura}

Um controlador de malha fechada proporcional interativo derivativo (PID) será utilizado para controlar a temperatura do fermentador. Esse método é amplamente utilizado na indústria, possuindo boa precisão e confiabilidade, além de ser facilmente sintonizado. Inicialmente são definidos parâmetros analógicos e depois é criado o controle digital.

\subsubsection{Controlador PID}

O controlador PID utiliza 3 ações (proporcional, integrativa e derivativa) controlar a planta minimizando o erro. Sua saída pode ser definida pela equação \ref{eq:pid}. A figura \ref{fig:pid} esquematiza a malha de controle.

\begin{equation}
    u(t) = K_pe(t) + K_i  \int_{0}^{t} e(\tau) \, d\tau + K_d\dfrac{de(t)}{\, dt}
    \label{eq:pid}
\end{equation}

Na qual:

\begin{table}[H]
    \begin{center}
        \begin{tabular}{ |c|c| } 
            \hline
            Símbolo   &  Descrição  \\
            \hline
            \(K_p\)   &  ganho proporcional  \\
            \hline
            \(K_i\)   &  ganho integrativo  \\
            \hline
            \(K_d\)   &  ganho derivativo  \\
            \hline
            \(e\)   &  erro  \\
            \hline
            \(t\)   &  tempo  \\
            \hline
            \(\tau\)   &  tempo de integração \\
            \hline
        \end{tabular}
        \caption{\label{tab:variaveis_pid}Descrição das variáveis do controlador PID da Equação \ref{eq:pid}.}
    \end{center}
\end{table}

\begin{figure}[h]
    \centering
    \includegraphics[scale=0.60]{figuras/implementacao/hardware/pid.jpg}
    \caption{Malha de controle PID.}
    \label{fig:pid}
\end{figure}


A ação proporcional do controlador produz um sinal de saída proporcional à amplitude do erro $e(t)$. Essa ação é útil para criar uma resposta equivalente ao tamanho do erro do sistema. A ação integral produz um sinal de saída proporcional à magnitude do erro, dependendo não só do seu valor, mas também da sua duração. Essa ação corrige o erro de off-set gerado pela ação proporcional e acelera a resposta do sistema. Por fim, a ação derivativa produz um sinal de saída proporcional à velocidade de variação do erro. Essa ação melhora a estabilidade e velocidade de resposta do sistema através de uma correção antecipada do erro. 


\subsubsection{Método de Ziegler-Nichols}


Para escolha dos parâmetros de ganho, o método de Ziegler-Nichols foi utilizado. Esse método foi escolhido principalmente pela sua simplicidade e facilidade na sintonização, visto que não é necessário uma modelagem matemática precisa da planta para o seu ajuste. 
A escolha dos parâmetros é feita a partir da observação da resposta do sistema em malha aberta com a aplicação de um degrau. A figura \ref{fig:resposta_sistema} ilustra a curva de resposta do sistema, e a tabela \ref{tab:parametros_ziegler} lista os parâmetros provenientes do método.


\begin{figure}[h]
    \centering
    \includegraphics[scale=0.45]{figuras/implementacao/hardware/resposta.png}
    \caption{Responda do Sistema em malha aberta a sinal degrau.}
    \label{fig:resposta_sistema}
\end{figure}

\begin{table}[H]
    \begin{center}
        \begin{tabular}{ |c|c|c|c| } 
            \hline
            Controlador & \(K_p\) & \(T_I\) & \(T_D\) \\
            \hline
            \(P\) & \(1/\alpha\) &  & \\
            \hline
            \(P + I\) & \(0,9/\alpha\) & \(3L\) & \\
            \hline
            \(P + I + D\) & \(1,2/\alpha\) & \(2L\) & \(0,5L\) \\
            \hline
        \end{tabular}
        \caption{\label{tab:parametros_ziegler} Parâmetros do controlador PID seguindo método de Ziegler-Nichols.}
    \end{center}
\end{table}


\subsubsection{Anti Wind-up}

Nesta aplicação, devido a limitações do atuador, é esperado uma saturação na saída. No caso onde o sistema não consegue chegar no setpoint, teremos uma situação chamada de wind-up, na qual, a sua resposta integral irá crescer de forma indefinida. Para evitar problemas no controlador, é necessário a implementação de um sistema anti wind-up, no caso, a ação escolhida será congelar a ação do controlador em caso de saturação.  



\section{Projeto de Hardware}

\subsection{Sensores e Atuadores}


\section{Projeto de Software}

A partir dos Requisitos Técnicos levantados, foi realizado o projeto de software. O projeto se iniciou com a definição de casos de uso a serem implementados, de forma a satisfazer os requisitos técnicos, em conjunto o diagrama de casos de uso definido pela UML foi elaborado para prover apresentação visual. Em seguida, as informações que devem ser armazenadas pelo sistema foram levantadas e a relação entre elas foi estabelecida, sendo representada no diagrama de entidade-relacionamento. O passo seguinte foi a definição da arquitetura de software do sistema.

No projeto a arquitetura de microsserviços foi escolhida como padrão para o sistema. Essa arquitetura define padrões de modularização do sistema em componentes pequenos e altamente especializados, conferindo facilidades de manutenção e escalabilidade em contraste com a arquitetura de monólito. Em contrapartida, o sistema é mais complexo de se implementar e implantar, devido a separação dos módulos, contudo, considerando o desejo de continuar este projeto após a entrega e os padrões de mercado atuais a abordagem de microsserviços é considerada a mais adequada. 

Definida a arquitetura, cada módulo do sistema foi especificado e foi elaborado o diagrama de componentes, definido pela UML, para ilustrar os pontos de comunicação e componentização da solução completa. Nessa etapa foram definidas as tecnologias a serem empregadas em cada módulo, a justificativa das escolhas é apresentada após o detalhamento dos componentes da arquitetura de software. O projeto de implantação foi então realizado, com planejamento da disponibilização do sistema de software com ferramentas de computação de nuvem disponíveis no mercado.

\subsection{Especificação dos Casos de Uso}

Na definição dos casos de uso do projeto foram definidos dois atores que interagem com o sistema de software a ser desenvolvido, identificados como 
usuário e dispositivo. O usuário representa o utilizador humano do sistema a ser desenvolvido, responsável por todas as interações humanas necessárias. 
O usuário se comunica com o sistema por duas interfaces, uma aplicação web, que se comunica diretamente com o sistema, e um aplicativo para smartphone, necessário para configurações iniciais do dispositivo. O dispositivo representa o sistema hardware de controle e monitoramento, também desenvolvido neste projeto. Em relação ao sistema de software, ele é tratado como um ator com suas devidas interações; seu projeto e especificações são discutidos na seção destinado ao projeto de hardware.

Segue a especificação dos casos de uso em si, contendo a identificação de cada caso de uso, sua breve descrição, enumeração dos passos que o definem e 
listagem dos requisitos técnicos relacionados ao caso de uso. Com caráter ilustrativo, o Diagrama de Casos de Uso da UML é apresentado na figura \ref{fig:diagrama_caso_de_usos}.

\subsubsection*{UC - 1: Registro de Dispositivo} 

Descrição: ao obter um novo dispositivo, o usuário deve configurar seu acesso à rede Wi-Fi e registrá-lo, de modo que o sistema reconheça que aquele 
dispositivo pertence ao usuário.
\begin{enumerate}
    \item Usuário acessa aplicativo em seu smartphone
    \item Sistema autentica acesso do usuário
    \item Aplicativo se conecta ao dispositivo
    \item Usuário informa configurações da rede Wi-Fi
    \item Aplicativo envia informações da rede para dispositivo
    \item Dispositivo se conecta na rede e se prepara para receber mensagens do sistema
    \item Aplicativo envia informações do dispositivo para o sistema
    \item Sistema cadastra informações do dispositivo e usuário
\end{enumerate}
Requisito relacionado: SW-F-12

\subsubsection*{UC - 2: Cadastro de Receitas}
Descrição: fluxo de cadastro de receitas.
\begin{enumerate}
    \item Usuário acessa tela de listagem de receitas
    \item Sistema exibe todas as receitas referentes ao usuário
    \item Usuário seleciona opção “Criar Receita” e acessa tela de cadastro de receita
    \item Usuário informa nome, estilo e observações da receita e clica em “Salvar”
    \item Sistema cadastra a receita no banco de dados
\end{enumerate}
Requisito relacionado: SW-F-4

\subsubsection*{UC - 3: Cadastro de Lotes}
Descrição: fluxo de cadastro de lotes, perfis de controle e associação de lote a um dispositivo.
\begin{enumerate}
    \item Usuário acessa tela de listagens de receitas
    \item Sistema exibe todas as receitas referentes ao usuário
    \item Usuário escolhe uma receita e seleciona a opção “Criar Lote”, e acessa a tela de cadastro de lote
    \item Sistema carrega listagem de perfis de controle já cadastrados e dispositivos do usuário
    \item Usuário informa identificação e observações do lote
    \item Usuário escolhe um perfil de controle já existente ou cria um novo perfil, informando uma identificação e cada um dos passos de controle (instante e valor alvo de temperatura)
    \item Usuário seleciona qual dispositivo irá controlar a produção do lote
    \item Usuário clica em “Salvar”
    \item Sistema cadastra lote, perfil de controle (caso novo), associação de lote e perfil de controle, e associação de lote e dispositivo
    \item Sistema envia informações do lote para dispositivo selecionado
\end{enumerate}
Requisitos relacionados: SW-F-5 e SW-F-6.

\subsubsection*{UC - 4: Envio das informações do Lote para Dispositivo}
Descrição: fluxo de envio das informações do lote para o dispositivo associado ao controle daquele lote.
\begin{enumerate}
    \item Dispositivo recebe informações do lote que foi associado por tópico de mensagens
    \item Dispositivo salva informações localmente
    \item Quando pronto, dispositivo inicia rotina de monitoramento e controle
\end{enumerate}    
Requisito relacionado: HW-F-5

\subsubsection*{UC - 5: Envio das informações do Dispositivo para o Sistema}
Descrição: fluxo de envio das informações obtidas pelo monitoramento do processo pelo dispositivo para o sistema.
\begin{enumerate}
    \item Dispositivo envia dados coletados para o sistema
    \item Sistema processa dados e salva informações no banco de dados
\end{enumerate}    
Requisitos relacionados: HW-F-6, SW-F-9

\subsubsection*{UC - 6: Visualização das Informações dos Lotes}
Descrição: fluxo para visualização das informações gerais e de evolução dos lotes correntes e passados
\begin{enumerate}
    \item Usuário acessa tela de listagem dos lotes
    \item Sistema exibe todas os lotes referentes ao usuário
    \item Usuário escolhe um lote e seleciona a opção “Ver Informações”, acessando a tela de informações do lote
    \item Sistema exibe informações gerais sobre o lote, como identificação, receita, observações, status, data de início, data de término, densidades relativas inicial e final/atual, estimativa de teor alcoólico, pH final/atual, temperatura final/atual.
    \item Sistema exibe um gráfico com as variáveis monitoradas em relação ao tempo
    \item Caso usuário clique em “Baixar Dados”, sistema efetua o download dos dados do gráfico em arquivo de texto
\end{enumerate}    
Requisitos relacionados: SW-F-2, SW-F-3, SW-F-8, SW-F-11

\begin{figure}[ht]
    \centering
    \includegraphics[scale=0.45]{figuras/projeto/software/diagrama_casos_de_uso.png}
    \caption{Diagrama de Casos de Uso.}
    \label{fig:diagrama_caso_de_usos}
\end{figure}

\subsection{Modelo de Entidade-Relacionamento}

O Modelo Entidade Relacionamento descreve como as informações são organizada no sistema e como o banco de dados do sistema é estruturado. 
Cada entidade no modelo representa um tipo de informação que é produzida e consumida pelo sistema, na execução dos casos de uso. 

Todas as informações são armazenadas em um banco de dados relacional, dividido em schemas que atuam como partições de dados em diferentes domínios. 
Foram definidos três domínios, traduzidos em schemas no banco de dados, para os dados: user, recipe e control. O schema user contém as informações 
referentes aos usuários e dispositivos do sistema; o domínio recipe contempla as entidades relacionadas às receitas e lotes de produção, assim como 
os dados coletados em cada execução de uma receita; e o schema control, por fim, os perfis de controle que são seguidos pelo dispositivo durante 
seu funcionamento. A separação das entidades em domínios é importante na arquitetura de microsserviços para assegurar que cada serviço tenha 
controle apenas às informações de sua competência. 

A definição das entidades e seus relacionamentos é ilustrada pelo Diagrama Entidade Relacionamento da figura \ref{fig:diagrama_entidade_relacionamento}, com destaque em cor para cada um 
dos schemas determinados.

\begin{figure}[ht]
    \centering
    \includegraphics[scale=0.15]{figuras/projeto/software/banco_de_dados.png}
    \caption{Diagrama de Entidade-Relacionamento.}
    \label{fig:diagrama_entidade_relacionamento}
\end{figure}

\subsection{Arquitetura de Software}

A arquitetura de software do projeto foi estruturada tendo como base o fluxo da informação como especificado nos casos de uso e os conceitos de arquitetura de microsserviços. Dessa forma, foram definidas quatro camadas para organizar os sistemas a serem implementados: Camada de Interface, contemplando as interfaces utilizadas diretamente pelos atores; Camada Intermediária, contendo um API Gateway para isolar os microsserviços das interfaces e um message broker para intermediar a comunicação entre dispositivo e sistema; Camada de Negócio, contendo os microsserviços que exercem as regras de negócio do sistema; e Camada de Persistência, com os serviços de armazenamento de dados. Os componentes de cada camada são descritos em detalhes quanto a suas responsabilidades e detalhes de implementação em sequência. A arquitetura completa é representada visualmente pelo Diagrama de Componentes da figura \ref{fig:diagrama_componentes}.

%% TODO: Diagrama complexo em apêndice

\subsubsection{Camada de Interface}
A Camada de Interface contém os componentes que fornecem interface direta aos atores definidos na modelagem de casos de uso. 

Aplicação Front End
    - Implementação: HTML + CSS + React.js
    - Responsabilidades: disponibilizar interfaces com o usuário para gerenciamento e visualização de suas informações

Aplicativo em SmartPhone
    - Implementação: aplicativo em Java para Android
    - Responsabilidades: envio de configurações iniciais ao dispositivo por bluetooth e registro do dispositivo no sistema

Dispositivo
    - Implementação: linguagem Arduino
    - Responsabilidades: monitoramento e controle do processo de fermentação

\subsubsection{Camada Intermediária}

    A camada intermediária acomoda um serviço de API Gateway, que age como um proxy reverso que roteia as requisições externas ao sistema para quais microsserviços sejam necessários e retorna ao cliente; e um serviço de Message Broker que controla o fluxo de mensagens que são transmitidas entre o dispositivo e o sistema.
    
    API Gateway
        - Implementação: aplicação flask (Python), servido com Gunicorn e Nginx
        - Responsabilidades: Centralizar as requisições ao sistema e traduzi-las em requisições ao microsserviços necessários para o completar a ação solicitada
    
    Message Broker
        - Implementação: Mosquitto (Protocolo MQTT)
        - Responsabilidades: Receber e disponibilizar mensagens em tópicos
    
\subsubsection{Camada de Negócio}
    
\subsubsection{Camada de Persistência}

\begin{figure}[ht]
    \centering
    \includegraphics[scale=0.50, angle=270]{figuras/projeto/software/diagrama_componentes.png}
    \caption{Diagrama de Componentes.}
    \label{fig:diagrama_componentes}
\end{figure}

\subsection{Definição de Tecnologias}

\subsection{Projeto de Implantação}

\subsection{Protótipos das Interfaces}
%% TODO: Apêndices




\chapter{Implementação}

% // TODO: Escrever Implementação (Software e Hardware)
\section{Implementação de Hardware}

\subsection{Sensor de Temperatura}

O sensor de temperatura DS18b20 é alimentado pelos pinos de 5V e GND (terra) do microcontrolador. seu sinal de leitura é enviado para uma porta digital, sendo utilizado um resistor de 4,7 k$\omega$ como pull up, caso exista uma leitura errada do sensor. A figura \ref{figura:micro_temp} ilustra a conexão do microcontrolador com o sensor de temperatura. 


\begin{figure}[h]
    \centering
    \includegraphics[scale=0.40]{figuras/implementacao/hardware/DSsensor.png}
    \caption{Esquema de conexão entre microcontrolador e sensor de temperatura DS18B20.}
    \label{fig:micro_temp}
\end{figure}


O excerto de código em linguagem Arduino a seguir representa a função utilizada para ler o valor de saída do sensor. A biblioteca OneWire implementa o protocolo proprietário de comunicação serial da Dallas Semicondutor, fabricante do sensor. Ela é utilizada em conjunto com a biblioteca livre DallasTemperature para estabelecer a comunicação com o DS18B20.

% // TODO: Referências das bibliotecas
% https://github.com/milesburton/Arduino-Temperature-Control-Library
% https://github.com/PaulStoffregen/OneWire

% \begin{lstlisting}[language=C]

% #include <OneWire.h>
% #include <DallasTemperature.h>
% #define ONE_WIRE_BUS D6

% OneWire oneWire(ONE_WIRE_BUS);
% DallasTemperature sensors(&oneWire)

% void setup() {
%     sensors.begin();
% }

% float readTemperature() {
%     return sensors.getTempCByIndex(0);
% }

% \end{lstlisting}


\subsection{Sensor de pH}

O sensor de pH é conectado a um circuito auxiliar que trata o sinal proveniente da ponta de prova. Esse circuito é conectado aos pinos de 5V e GND do microcontrolador para alimentação, e envia do dado coletado por meio de sinal analógico, como observado na figura \ref{fig:micro_ph}. 


\begin{figure}[h]
    \centering
    \includegraphics[scale=0.65]{figuras/implementacao/hardware/micro_ph.jpg}
    \caption{Esquema de conexão entre microcontrolador e sensor de pH E-201-C.}
    \label{fig:micro_ph}
\end{figure}


Para utilização desse sensor, é necessário sua calibração, que consiste na medição da tensão de saída do sensor em soluções com diferentes concentrações de \(H^+\) e obtenção de uma equação linear que traduza a voltagem medida em um valor de pH. Para isso utilizadas soluções tampão com pH iguais a 4, 7 e 10. A calibração seguiu o seguinte processo:
\begin{enumerate}
    \item Com a solução de pH igual a 7, ajustar o ganho do circuito (por um potenciômetro) até que a leitura de voltagem seja 2,5 Volts. Essa tensão foi escolhida de forma a coincidir valor médio da tensão de alimentação do circuito com o valor médio da escala de pH;
    \item Capturar a tensão medida nas soluções com pH 4 e 10;
    \item Aplicar regressão linear, minimizando a soma dos erros quadráticos, obtendo a reta que relaciona a tensão medida com o pH da solução
\end{enumerate}


Com os valores obtidos na tabela \ref{tab:calibra_ph}, foi obtida a equação linear \ref{eq:pH}.

\begin{equation}
    pH = 2.0171 * Voltagem + 1.7152
    \label{eq:pH}
\end{equation}


\begin{table}[H]
    \begin{center}
        \begin{tabular}{ |c|c| } 
            \hline
            Voltagem & pH \\
            \hline
            2.50 & 7.0 \\ 
            \hline
            1.20 & 4.0 \\ 
            \hline
            4.16 & 10.0 \\ 
            \hline
        \end{tabular}
        \caption{\label{tab:calibra_ph}Medições de tensão em soluções tampão para calibração do sensor de pH.}
    \end{center}
\end{table}


A leitura do pH medido pelo sensor pelo microcontrolador segue o seguinte algoritmo:
\begin{enumerate}
    \item Valor inteiro entre 0 e 1024 recebido pelo sensor é lido na porta analógica;
    \item Valor lido é convertido em uma diferença de tensão seguindo \(Voltagem = medida * 5 / 1024\);
    \item Diferença de tensão é convertida na leitura de pH seguindo equação linear obtida na calibração do sensor.
\end{enumerate}


% \begin{lstlisting}[language=C]

% int ph_pin = A0;

% float readPH() {
%     int measure = analogRead(ph_pin);
%     double voltage = measure * 5 / 1024;
%     return 2.0171 * voltage + 1.7152;
% }

% \end{lstlisting}


\subsection{Conexão e Comunicação do Microcontrolador}

Com os sensores propriamente conectados e calibrados, o microcontrolador foi configurado para se conectar à Internet através de conexão WiFi e a enviar os dados coletados por meio do protocolo MQTT. O Wemos D1 é controlada pelo módulo ESP8266, de forma que ofereça conectividade WiFi nativa. 
A biblioteca ESP8266WiFi foi utilizada para estabelecer a conexão Wifi, em conjunto com a biblioteca PubSubClient para publicar mensagens e se inscrever em tópicos MQTT. 

% https://github.com/esp8266/Arduino/tree/master/libraries/ESP8266WiFi
% https://github.com/knolleary/pubsubclient

% // TODO: Colocar código completo em Apêndice e discutir Fluxo completo aqui

\subsection{Controle de Temperatura}

Um controlador de malha fechada proporcional interativo derivativo (PID) será utilizado para controlar a temperatura do fermentador. Esse método é amplamente utilizado na indústria, possuindo boa precisão e confiabilidade, além de ser facilmente sintonizado. Inicialmente são definidos parâmetros analógicos e depois é criado o controle digital.


\subsubsection{Controlador PID}

O controlador PID utiliza 3 ações (proporcional, integrativa e derivativa) controlar a planta minimizando o erro. Sua saída pode ser definida pela equação \ref{eq:pid}. A figura \ref{fig:pid} esquematiza a malha de controle.

\begin{equation}
%     u(t) = K_pe(t) + K_i\[\int_{0}^{t} e(\tau) \, d\tau \] + K_d\dfrac{de(t)}{dt}
    u(t) = K_pe(t) + K_i  \int_{0}^{t} e(\tau) \, d\tau + K_d\dfrac{de(t)}{\, dt}
    \label{eq:pid}
\end{equation}

Na qual:

\begin{table}[H]
    \begin{center}
        \begin{tabular}{ |c|c| } 
            \hline
            Símbolo   &  Descrição  \\
            \hline
            \(K_p\)   &  ganho proporcional  \\
            \hline
            \(K_i\)   &  ganho integrativo  \\
            \hline
            \(K_d\)   &  ganho derivativo  \\
            \hline
            \(e\)   &  erro  \\
            \hline
            \(t\)   &  tempo  \\
            \hline
            \(\tau\)   &  tempo de integração \\
            \hline
        \end{tabular}
        \caption{\label{tab:variaveis_pid}Descrição das variáveis do controlador PID da Equação \ref{eq:pid}.}
    \end{center}
\end{table}

\begin{figure}[h]
    \centering
    \includegraphics[scale=0.60]{figuras/implementacao/hardware/pid.jpg}
    \caption{Malha de controle PID.}
    \label{fig:pid}
\end{figure}


A ação proporcional do controlador produz um sinal de saída proporcional à amplitude do erro $e(t)$. Essa ação é útil para criar uma resposta equivalente ao tamanho do erro do sistema. A ação integral produz um sinal de saída proporcional à magnitude do erro, dependendo não só do seu valor, mas também da sua duração. Essa ação corrige o erro de off-set gerado pela ação proporcional e acelera a resposta do sistema. Por fim, a ação derivativa produz um sinal de saída proporcional à velocidade de variação do erro. Essa ação melhora a estabilidade e velocidade de resposta do sistema através de uma correção antecipada do erro. 


\subsubsection{Método de Ziegler-Nichols}


Para escolha dos parâmetros de ganho, o método de Ziegler-Nichols foi utilizado. Esse método foi escolhido principalmente pela sua simplicidade e facilidade na sintonização, visto que não é necessário uma modelagem matemática precisa da planta para o seu ajuste. 
A escolha dos parâmetros é feita a partir da observação da resposta do sistema em malha aberta com a aplicação de um degrau. A figura \ref{fig:resposta_sistema} ilustra a curva de resposta do sistema, e a tabela \ref{tab:parametros_ziegler} lista os parâmetros provenientes do método.


\begin{figure}[h]
    \centering
    \includegraphics[scale=0.45]{figuras/implementacao/hardware/resposta.png}
    \caption{Responda do Sistema em malha aberta a sinal degrau.}
    \label{fig:resposta_sistema}
\end{figure}

\begin{table}[H]
    \begin{center}
        \begin{tabular}{ |c|c|c|c| } 
            \hline
            Controlador & \(K_p\) & \(T_I\) & \(T_D\) \\
            \hline
            \(P\) & \(1/\alpha\) &  & \\
            \hline
            \(P + I\) & \(0,9/\alpha\) & \(3L\) & \\
            \hline
            \(P + I + D\) & \(1,2/\alpha\) & \(2L\) & \(0,5L\) \\
            \hline
        \end{tabular}
        \caption{\label{tab:parametros_ziegler} Parâmetros do controlador PID seguindo método de Ziegler-Nichols.}
    \end{center}
\end{table}



\subsubsection{Anti Wind-up}


Nesta aplicação, devido a limitações do atuador, é esperado uma saturação na saída. No caso onde o sistema não consegue chegar no setpoint, teremos uma situação chamada de wind-up, na qual, a sua resposta integral irá crescer de forma indefinida. Para evitar problemas no controlador, é necessário a implementação de um sistema anti wind-up, no caso, a ação escolhida será congelar a ação do controlador em caso de saturação.  




\chapter{Testes e Avaliação do Protótipo}

O intuito original dos testes de validação era aferir o funcionamento do protótipo durante uma fermentação real, 
realizada em pequena escala exclusivamente para avaliação desse projeto. Contudo, devido a atrasos no cronograma
nas últimas etapas de montagem do protótipo e documentação, não foi possível preparar esse teste. Decorrente disso,
o protótipo foi testado em situações não ideais, mas ainda assim relativamente semelhantes ao processo real, e os
dados coletados foram observados na Interface Web e comparados com valores inferidos manualmente, 

% \section{Avaliação do Hardware}

Primeiramente foram testados os sensores do protótipo. Para o sensor de temperatura, utilizou-se um recipiente com água
comparando a temperatura lida em diferentes temperaturas com um termômetro analógico. Considerando a resolução da escala do
termômetro analógico de 1°C e a resolução de duas casas decimais do sinal digital, todas as medidas feitas estavam de acordo
e dentro da margem de erro. O sensor de pH foi validado após a calibração, usando as soluções tampão de pH 4, 7, e 10, com erro
máximo obtido de 0,1 unidade de medição, o que é suficiente para a aplicação. 

Por fim, o sensor de densidade foi testado em água,
mantendo uma distância de 18 cm entre os dois pontos de medição, e comparado com o resultado de um densímetro analógico,
obtendo um erro de aproximadamente 0,3\%, que foi bem abaixo do valor esperado, dado a acurácia nominal de 5\% do sensor, 
seria interessante coletar mais medições com diferentes fluidos para melhor aferimento do erro do sensor. Caso o sensor apresentasse
um resultado pouco confiável, ele poderia ser usado apenas para conferir uma noção qualitativa de evolução do processo pela 
tendência de mudança da densidade, porém, ele aparenta ser adequado para cálculo de índices relevantes ao processo como 
gradação alcoólica, eficiência e atenuação, que se baseiam nos valores de densidade relativa antes e depois da fermentação.

O atuador de temperatura, composto pela pastilha de Peltier com o dissipador de calor e barra de aço inoxidável para troca de
calor com o mosto foi também testado em água, utilizando um fermentador de plástico com capacidade de 20 litros. Foi montado
um isolamento simples ao entorno do fermentador, utilizando poliestireno expandido (comumente denominado "isopor") e espuma
vinílica acetinada (EVA). A partir da temperatura ambiente de aproximadamente 23°C, o atuador foi ligado em potência máxima
para medir a potência real de retirada de calor do sistema, obtendo-se algo próximo de 0,3°C por hora, para resfriar os 20 litros
de água. Acredita-se que esse desempenho é insuficiente, e que um melhor isolamento térmico e possíveis melhorias na montagem
do sistema de troca de calor possam melhorá-lo. Contudo, não se avançou nesse aspecto.


% \section{Avaliação do Software}



\chapter{Conclusão}

\section{Considerações Finais}

Ao final do presente trabalho, tem-se em mãos um dispositivo que coleta dados de temperatura, densidade e pH e os envia 
para um servidor online e que aplica um algoritmo de controle sobre a temperatura, e também um sistema de software 
que recebe os dados coletados e os exibe em poucos segundos em um gráfico online. 
Apesar do controle ainda necessitar de melhorias de montagem e isolamento térmico para atingir a potência esperada, e o 
software desenvolvido não ter atendido completamente o processo, como implementação de controle de acesso e o fluxo para 
configuração do dispositivo previsto nos casos de uso; acredita-se que foi desenvolvido um protótipo viável com um caminho evidente de melhorias.

Contudo, lamenta-se de não se ter dedicado o empenho desejado na etapa de teste e validação do protótipo. Essa etapa
foi prejudica por atrasos no cronograma, principalmente no momento intermediário do projeto. Com certeza, um controle mais rígido
das tarefas e disciplina com o cronograma teria afetado positivamente o projeto, especialmente nessa etapa.

Por outro lado, as etapas de projeto e prototipação em etapas foram consideravelmente proveitosas como uma aplicação
prática e extensa dos conteúdos que foram estudados durante a graduação. Ter atuado nesse trabalho, que se iniciou com
a busca de entendimento sobre um processo, seguindo por levantamento de idéias de melhoria e projeto de uma solução e 
finalizando pela implementação configurou um grande aprendizado de engenharia.

No mais, considerando o objetivo do projeto de interesse pessoal dos autores, espera-se dar continuidade ao trabalho após
a entrega desse documento, e que esse relatório possa estimular outros alunos interessados e ajudá-los na 
elaboração de seus projetos.

\section{Perspectivas de Continuidade}

Assim como foi mencionado, foram notadas nesse documento algumas melhorias que podem ser feitas sobre o escopo do trabalho atual,
como melhorar a aplicação do controle de temperatura, refinar a montagem do dispositivo, finalizar a implementação da Interface
Web com controle de acessos e dados segregados por usuário, e refinar as interações entre hardware e software focando nos requisitos
não funcionais levantados. Acredita-se que com poucas semanas adicionais, boa parte disso poderia ser realizado.

Sob a ótica acadêmica, uma continuação interessante ao trabalho seria estudar a viabilidade de estruturar o algoritmo de controle
levando em consideração a densidade do mosto em fermentação. Isso poderia permitir inclusive reduzir o tempo de fermentação em
alguns dias, como discutido brevemente por \citeonline{YeastWhite}, e possui grande valor para cervejarias, mesmo de grande porte 
que poderiam adaptar a estratégia a sua escala. Contudo, para essa abordagem, acredita-se ser necessário um maior compreendimento do 
processo bioquímico e da natureza das leveduras.

Sob a ótica de mercado, aplicando-se um refinamento sobre o protótipo deve ser possível transformá-lo em um produto, assim como
foi imaginado no início desse projeto. Os devidos estudo sobre mercado deveriam ser realizados, e seria interessante mais entrevistas 
com produtores para validar o produto, mas acredita-se sê-lo viável, ainda que para um mercado de nicho.


% \section{Contribuições}
% \textcolor{red}{\{Apresentar as contribuições  trabalho, ressaltando o que foi autoria do grupo\}}


% ========== Referências ==========
% --- IEEE ---
%	http://www.ctan.org/tex-archive/macros/latex/contrib/IEEEtran
%\bibliographystyle{IEEEbib}

% --- ABNT (requer ABNTeX 2) ---
%	http://www.ctan.org/tex-archive/macros/latex/contrib/abntex2
\bibliographystyle{abntex2-alf}
\bibliography{./referencias/bibliografia} %Localização do arquivo bibliografia.bib
%\bibliography{}


% ========== Apêndices (opcional) ==========
\apendice{}
\chapter{Levantamento Bibliográfico}

% // TODO: Escrever Levantamento Bibliográfico (Exemplo TCC Franco)

\label{appendice:levantamento_bibliografico}

\chapter{Código fonte}

Os códigos fonte do projeto foram hospedados em repositórios virtuais na plataforma Github. O link de acesso a cada repositório é listado a seguir.


% //TODO: Colocar repositórios restantes

\textbf{Aplicação Web} - Front-End do projeto: \url{https://github.com/josehcls/tcc-front-end}


\textbf{Dispositivo} - Software embarcado do dispositivo: \url{https://github.com/josehcls/tcc-device}


\textbf{Servidor de Aplicação} - Serviço de API Gateway: \url{https://github.com/josehcls/tcc-app}


\textbf{Servidor de Registro} - Serviços de Usuário e Registro: \url{https://github.com/josehcls/tcc-register-service}


\textbf{Servidor de Análises} - Serviço de Análises: \url{-}


\textbf{Servidor de Receitas} - Serviços de Receitas, Lotes e Controle: \url{https://github.com/josehcls/tcc-recipe-service}


\textbf{Servidor de Processamento} - Serviço Processador: \url{https://github.com/josehcls/tcc-processor}


\textbf{Configuração} - Arquivos gerais, como Docker-compose para implantar os servidores e DDL do Banco de Dados: \url{https://github.com/josehcls/tcc-config}


\textbf{Monografia} - Arquivos utilizados na geração desta monografia (LaTeX): \url{https://github.com/josehcls/tcc-monografia}


\label{appendice:repositorios}

% // TODO: Links para Repositórios

% ========== Anexos (opcional) ==========
\anexo
% \chapter{}



\end{document}
