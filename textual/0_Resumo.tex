Na produção de cervejas, a fermentação é um complexo processo bioquímico cuja função primária é converter os açúcares obtidos dos grãos maltados em etanol. Esse processo tem grande impacto no sabor, aroma, aparência e textura do produto, sendo também um dos mais negligenciados durante a produção de cervejas em pequena escala. 
As principais variáveis desse processo são: a densidade do líquido, que indica a evolução da fermentação, a temperatura do líquido, que impacta diretamente no metabolismo da levedura, e o pH, que é um indicador adicional de qualidade. 
O objetivo geral desse projeto é criar um protótipo de um sistema que realize o monitoramento dessas variáveis e controle a temperatura, de modo a garantir resultados mais precisos e reprodutíveis. O sistema é voltado para cervejarias de pequeno e médio porte que realizem testes de novas receitas e necessitem de alto grau de controle para escalas pequenas de produção. 
O sistema idealizado é composto por dois principais subsistemas: (i) um físico, agregando o controlador, hardware com sensores e atuadores e software embarcado; e (ii) um digital, que capta, processa e disponibiliza todas as informações adquiridas para o usuário. A execução do projeto será guiada por várias iterações curtas e prototipagem, focando na resolução de algumas milestones a cada iteração, com a finalidade de acelerar a obtenção de resultados.
Deste modo, para a aquisição de conhecimento técnico e de negócio necessários para a execução do projeto foi realizado um levantamento e estudo de material bibliográfico adequado, entrevistas com mestres-cervejeiros para o qual o produto se destinaria e análise de soluções já existentes no mercado, dentro e fora do país.
