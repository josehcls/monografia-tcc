Na produção de cervejas, a fermentação alcoólica é um complexo bioquímico cuja função primária é converter os açúcares extraídos dos grãos maltados em álcool etílico. Esse processo tem grande impacto no sabor, aroma, aparência e textura do produto, sendo o processo mais importante durante a produção da bebida. 
As principais variáveis desse processo são: a temperatura, que influencia no metabolismo das leveduras, a densidade, que indica a evolução da fermentação, e o pH, que é um indicador adicional de qualidade. 
O objetivo desse projeto é criar um protótipo de um dispositivo que realize o monitoramento dessas variáveis e controle a temperatura, garantindo resultados mais precisos e reprodutíveis. O sistema é voltado para cervejarias de pequeno e médio porte que desejem monitorar e controlar o processo durante os testes de receitas em pequena escalas de produção, e também para produtores \textit{hobbyistas} que almejem maior controle sobre o processo e resultados mais consistentes. 
O sistema idealizado é composto por dois subsistemas complementares: (i) um físico: agregando o controlador, hardware com sensores e atuadores e software embarcado; e (ii) um digital: que capta, processa e disponibiliza todas as informações adquiridas para o usuário por meio de uma plataforma online. A execução do projeto foi guiada por iterações curtas e prototipagem, focando na implementação gradativa de funcionalidades a cada iteração até a construção total do protótipo.
 % Deste modo, para a aquisição de conhecimento técnico e de negócio necessários para a execução do projeto foi realizado um levantamento e estudo de material bibliográfico adequado, entrevistas com mestres-cervejeiros para o qual o produto se destinaria e análise de soluções já existentes no mercado, dentro e fora do país.