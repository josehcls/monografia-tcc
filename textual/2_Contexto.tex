\chapter{Contexto}

Esse projeto tem natureza multidisciplinar, buscando harmonizar a bioquímica do
processo de produção de cervejas com sistemas de controle e automação e
geração e análise de dados estudados na Engenharia Elétrica. Em sua primeira
parte, será estudado o funcionamento das leveduras e como o ambiente age sobre
seu processo metabólico, enquanto na outra parte, são estudados sensores,
atuadores e arquitetura de Internet das Coisas para captação, processamento e
disponibilização de informações referente a ação dos microrganismos.

\section{História da Cerveja}

% // TODO: Colocar fontes bibliográficas
\textcolor{red}{\{Falta citar as fontes bibliográficas\}}

A cerveja pode ser considerada a bebida alcoólica mais antiga do mundo, e atualmente tem uma importância social e econômica muito grande. Durante anos, a sua produção ganhou diversas mudanças, melhorias e adaptações até chegar nas variedades de forma e sabor conhecidas atualmente. 


A cerveja provavelmente teve origem na revolução agrícola, na qual os humanos começaram a abandonar o nomadismo e se estabeleceram em comunidades que cultivam diversos grãos. Com o armazenamento de cereais, é provável que essa origem tenha sido acidental, com uma fermentação espontânea da cevada. Os mais antigos registros dessa bebida podem ser encontrados na antiga região mesopotâmia, atual Irã, e são datados em 2800 a.C. A sua importância em comunidades antigas do oriente médio era tanta, que em 1760 a.C., foi criada a primeira lei que regulamenta a produção e venda de cerveja. A lei conhecida como Estela de Hamurabi regulamentou a comercialização, fabricação e consumo, estabelecendo uma ração diária de cerveja para os habitantes da região. 


Como o consumo da cerveja era mais seguro do que a água, visto que o processo de fervura ajuda a purificar a bebida. O  seu consumo em algumas sociedades era visto como uma necessidade básica diária. Esse fato ajudou a aumentar a popularidade da bebida na Europa durante a idade média, principalmente nas comunidades Germânicas. Uma das instituições mais importantes para o desenvolvimento dos processos de produção eram os mosteiros dessa região e época. Os monges eram responsáveis pela fabricação da bebida e como eram os únicos que reproduziam os manuscritos, puderam conservar e aperfeiçoar a sua produção, sendo muito influentes até hoje. Eles foram responsáveis por incluir diversas ervas na fabricação, sendo a mais importante delas o lúpulo, utilizado para trazer o amargor da bebida. 


Uma das leis mais conhecidas e importantes da indústria cervejeira é a da pureza Alemã. Devido a diversas mudanças aplicadas pelos fabricantes e a percepção de uma queda de qualidade, essa lei foi criada no século XIV na região da Bavária (Sul da Alemanha) buscando uma padronização da bebida. A lei instituiu que a cerveja deveria ser fabricada apenas com os seguintes ingredientes: água, malte de cevada e lúpulo. Atualmente, apesar da legalização do uso de qualquer ingrediente na região, muitos cervejeiros ainda seguem essa restrição e ela é um sinônimo de qualidade. 


Uma das mais importantes inovações na fabricação foi a Pasteurização, que permite a preservação do gosto por mais tempo. O processo consiste, basicamente, no aquecimento da bebida a uma determinada temperatura, por determinado tempo, e depois a bebida é resfriada de forma a eliminar os microorganismos ali presentes. O processo foi nomeado pelo seu criador o francês Louis Pasteur que atendendo a solicitação de alguns dos vinicultores e cervejeiros da região que lhe pediram para descobrir como os vinhos e a cervejas azedaram. Durante sua investigação, através do uso de microscópio, ele pôde constatar que a levedura ocasionava este processo e assim criou esse processo de purificação. A partir dele, a indústria cervejeira conseguiu chegar em um novo nível e crescer em escala e alcance, sendo essencial para grandes fábricas atualmente. 


Atualmente, a indústria de cerveja movimenta milhões de dólares anualmente e uma das divisões que mais cresce são as microcervejarias. No Brasil começaram esses pequenos produtores começaram a surgir na década de 90. Em 2012 as cervejas especiais representavam 8\% do mercado nacional da bebida em 2012 e encerraram 2014 com uma participação de 11\%, segundo o Sindicato Nacional da Indústria da Cerveja, que aponta a existência de 300 microcervejarias no País. A projeção é de que essa cota suba para 20\% em 2020.


\section{Fabricação de Cervejas}

\citeonline{Lewis} apresenta a produção de cerveja como uma atividade que deve equilibrar 
séculos de tradição e arte, desenvolvida por gerações de cervejeiros, e a abordagem científica 
e avanços tecnológicos, de forma que estes possam proporcionar maior entendimento, controle e melhorias 
sobre o processo mas sem abandonar suar raízes históricas e descaracterizá-lo. 
Justamente, as etapas principais do processo ainda seguem a produção tradicional, apesar
da evolução de métodos e técnicas adicionais.

Nesta seção é descrito o processo mais comum de produção de cervejas, considerando os principais
ingredientes: água, malte, lúpulo e levedura. Tradicionalmente, são utilizados grãos maltados de cevada,
mas atualmente podem ser utilizados outros grãos, maltados ou não, como por exemplo: trigo, milho, arroz e aveia.
A partir da descrição de \citeonline{Kunze}, listamos a essência das principais etapas de produção:


\begin{enumerate}
    \item Maltagem dos grãos: processo de germinação parcial e controlada dos grãos, com a finalidade
de produzir enzimas como a amilase. A germinação é interrompida por um processo de secagem quando atinge o estágio desejado.
A partir desse momento, os grãos passam a ser denominados malte;
    \item Moagem do malte: quebra do malte em pequenos fragmentos para expor as enzimas e componentes internos. É desejável
que parte da casca seja mantida intacta para auxiliar na filtragem após a mostura;
    \item Mostura: o malte moído é misturado em água, e aquecida em temperaturas
que estimulem a ação das enzimas obtidas na maltagem. As enzimas
realizam a quebra de moléculas insolúveis de amido em moléculas menores
de açúcares, que são dissolvidas, formando uma solução denominada mosto;
    \item \textit{Lautering}: o mosto é separado do restante do malte que não foi dissolvido. As
cascas dos grãos auxiliam essa etapa formando um filtro natural, mas também podem ser utilizados filtros;
    \item Fervura: o mosto é fervido, consequentemente: a ação enzimática é
interrompida e a solução é esterilizada. Nessa etapa, lúpulos são adicionados em diferentes
momentos da fervura, fornecendo extratos que conferem amargor e aroma à cerveja;
    \item Fermentação: após o resfriamento do mosto, leveduras são adicionadas e a solução é aerada. 
Os açúcares são consumidos pelo metabolismo das leveduras, gerando etanol e dióxido de carbono, e alguns subprodutos;
    \item Maturação: após o consumo dos açúcares, as leveduras passam a reabsorver parte dos
subprodutos da fermentação, melhorando a qualidade da bebida. Ao final do processo, as leveduras floculam e decantam, podendo
ser facilmente extraídas e reaproveitadas;
    \item Envase: transferência para o recipiente final. Nessa etapa, também é
realizada a carbonatação da bebida, geralmente por injeção de gás carbônico
ou por meio de uma segunda fermentação, aproveitando-se leveduras ainda presentes na solução.
Industrialmente, é comum a filtragem pré-envase, e a pasteurização pós-
envase.
\end{enumerate}

\subsection{Fermentação e Maturação}

% // TODO: Melhorar Fermentação e Maturação
\textcolor{red}{\{Acho que vale explicarmos com um pouco mais de detalhe o processo de fermentação aqui, e como variáveis como temperatura, pH, oxigenação influenciam sobre ele.\}}

Segundo descrito por \citeonline{YeastWhite}, as primeiras produções de
cervejas datam de milhares de anos atrás, sendo que na maior parte da história a
fermentação de bebidas foi tida como um fenômeno divino, sem o conhecimento dos
organismos microscópicos que a realiza. Estes só puderam ser observados em 1680
com o desenvolvimento e auxílio de microscópios. Ainda assim, apenas em 1789 a
equação química da transformação de açúcares em álcool, dentre eles o etanol, e
dióxido de carbono ($\mathrm{CO_2}$) foi descrita por Lavoisier, e na segunda metade do século
XIX, com as descobertas de Pasteur, passaram a ser melhor entendidas, com o
surgimento da bioquímica como uma área de estudo própria. 


Apesar da falta de conhecimento durante tantos anos, a fermentação é o processo
que possui um dos maiores impactos no sabor, aroma, aparência e textura do
produto. Sendo assim, o controle dessa etapa é essencial para garantir a
qualidade final.  


A fermentação, na produção de cervejas, é um processo metabólico realizado pelas
leveduras adicionadas ao mosto com a finalidade principal de converter os açúcares
extraídos dos grãos maltados no processo de brassagem em etanol. As leveduras,
fungos unicelulares, realizam a fermentação como uma forma de
respiração anaeróbia para obterem energia em meios desprovidos de oxigênio ou
com grande excesso de açúcares disponíveis. Além da geração de etanol e gás
carbônico, há diversos subprodutos, como ésteres, álcoois superiores e compostos
sulfúricos que caracterizam o produto e definem sua qualidade final. 


O processo de fermentação pode ser dividido em 3 fases:  
\begin{enumerate}
    \item Fase de retardamento, durante as primeiras quatro a 15 horas após a adição
da levedura no mosto, caracterizada pela climatização das células ao
ambiente e preparação para a próxima fase;
    \item Fase de crescimento exponencial, que dura entre quatro horas e quatro dias,
quando ocorre o consumo dos açúcares e replicação logarítmica das células
de leveduras;
    \item Fase estacionária, em que o crescimento diminui e alguns compostos são
absorvidos pelas células, maturando o produto durante três a dez dias.
\end{enumerate}


Para a obtenção de resultados consistentes em testes e na produção de cervejas, é
essencial que a fermentação seja monitorada e controlada de forma que ocorra em
condições ideais. As principais variáveis monitoradas no processo, em ordem de
importância são: a temperatura, que impacta diretamente no grau de atividade
celular das leveduras, assim como na produção e reabsorção de subprodutos da
fermentação; a densidade relativa (specific gravity, em inglês) do mosto em
fermentação, que indica a evolução do processo, uma vez que ao longo do processo
a densidade diminui, em caminho a um valor final esperado; o pH da solução,
importante para a saúde da levedura e indicativo de possíveis problemas; e a
concentração de oxigênio e gás carbônico dissolvidos. Munroe (\citeonline{FermentationMunroe}) fornece um
perfil de evolução de algumas dessas variáveis na fermentação de uma cerveja do
tipo Lager (Figura \ref{fig:variaveis_fermentacao}):

\begin{figure}[H]
    \centering
    \includegraphics[scale=0.40]{figuras/contexto/variaveis_fermentacao.PNG}
    \captionsource{Gráfico de evolução de variáveis ao longo da fermentação.}{\citeonline{FermentationMunroe}}
    \label{fig:variaveis_fermentacao}
\end{figure}

\section{Estado da Arte de Sistemas de Produção Artesanal}

\subsection{Análise de Soluções Existentes}

% // TODO: Escrever Estado da Arte e Análise de Soluções Existentes
\textcolor{red}{\{Uma breve listagem de soluções e produtos existentes que encontramos em nossas pesquisas, considerando como é feito em grande e pequena escala.\}}
