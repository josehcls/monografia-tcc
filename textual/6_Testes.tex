\chapter{Testes e Avaliação do Protótipo}

O intuito original dos testes de validação era aferir o funcionamento do protótipo durante uma fermentação real, 
realizada em pequena escala exclusivamente para avaliação desse projeto. Contudo, devido a atrasos no cronograma
nas últimas etapas de montagem do protótipo e documentação, não foi possível preparar esse teste. Decorrente disso,
o protótipo foi testado em situações não ideais, mas ainda assim relativamente semelhantes ao processo real, e os
dados coletados foram observados na Interface Web e comparados com valores inferidos manualmente, 

% \section{Avaliação do Hardware}

Primeiramente foram testados os sensores do protótipo. Para o sensor de temperatura, utilizou-se um recipiente com água
comparando a temperatura lida em diferentes temperaturas com um termômetro analógico. Considerando a resolução da escala do
termômetro analógico de 1°C e a resolução de duas casas decimais do sinal digital, todas as medidas feitas estavam de acordo
e dentro da margem de erro. O sensor de pH foi validado após a calibração, usando as soluções tampão de pH 4, 7, e 10, com erro
máximo obtido de 0,1 unidade de medição, o que é suficiente para a aplicação. 

Por fim, o sensor de densidade foi testado em água,
mantendo uma distância de 18 cm entre os dois pontos de medição, e comparado com o resultado de um densímetro analógico,
obtendo um erro de aproximadamente 0,3\%, que foi bem abaixo do valor esperado, dado a acurácia nominal de 5\% do sensor, 
seria interessante coletar mais medições com diferentes fluidos para melhor aferimento do erro do sensor. Caso o sensor apresentasse
um resultado pouco confiável, ele poderia ser usado apenas para conferir uma noção qualitativa de evolução do processo pela 
tendência de mudança da densidade, porém, ele aparenta ser adequado para cálculo de índices relevantes ao processo como 
gradação alcoólica, eficiência e atenuação, que se baseiam nos valores de densidade relativa antes e depois da fermentação.

O atuador de temperatura, composto pela pastilha de Peltier com o dissipador de calor e barra de aço inoxidável para troca de
calor com o mosto foi também testado em água, utilizando um fermentador de plástico com capacidade de 20 litros. Foi montado
um isolamento simples ao entorno do fermentador, utilizando poliestireno expandido (comumente denominado "isopor") e espuma
vinílica acetinada (EVA). A partir da temperatura ambiente de aproximadamente 23°C, o atuador foi ligado em potência máxima
para medir a potência real de retirada de calor do sistema, obtendo-se algo próximo de 0,3°C por hora, para resfriar os 20 litros
de água. Acredita-se que esse desempenho é insuficiente, e que um melhor isolamento térmico e possíveis melhorias na montagem
do sistema de troca de calor possam melhorá-lo. Contudo, não se avançou nesse aspecto.


% \section{Avaliação do Software}

