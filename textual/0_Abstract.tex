During beer production, the alcoholic fermentation is a complex biochemical process, which primary function is to convert the sugar, extracted from the malted grains, in ethyl alcohol. This process has great impact on the flavor, scent, aspect and texture of the product, being the most important process in beer production.
The main variables during the process are: temperature, which impacts on yeast metabolism, specific gravity, which indicates fermentation evolution, and pH, a extra quality indicator.
The object of this project is to develop a prototype able to monitor these variables and control the temperature, allowing more precise and reproducible results. The system is target to small and medium breweries that desires fermentation monitoring and control during small scale recipes tests, and also to homebrewers yearning a better control over the process and more consistent results.
The system is composed by two complementary subsystems: (i) a physical one, consisting in the controller, hardware with sensors and actuator and embedded software; and a digital one which receive, process and presents all the data acquired in a web application to the user.
The project execution was guided by short iterations and prototyping, aiming at continuous feature development until the final version of the prototype.