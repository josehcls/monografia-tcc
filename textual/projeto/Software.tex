\section{Projeto de Software}

A partir dos Requisitos Técnicos levantados, foi realizado o projeto de software. O projeto se iniciou com a definição de casos de uso a serem implementados, de forma a satisfazer os requisitos técnicos, em conjunto o diagrama de casos de uso definido pela UML foi elaborado para prover apresentação visual. Em seguida, as informações que devem ser armazenadas pelo sistema foram levantadas e a relação entre elas foi estabelecida, sendo representada no diagrama de entidade-relacionamento. O passo seguinte foi a definição da arquitetura de software do sistema.

No projeto a arquitetura de microsserviços foi escolhida como padrão para o sistema. Essa arquitetura define padrões de modularização do sistema em componentes pequenos e altamente especializados, conferindo facilidades de manutenção e escalabilidade em contraste com a arquitetura de monólito. Em contrapartida, o sistema é mais complexo de se implementar e implantar, devido a separação dos módulos, contudo, considerando o desejo de continuar este projeto após a entrega e os padrões de mercado atuais a abordagem de microsserviços é considerada a mais adequada. 

Definida a arquitetura, cada módulo do sistema foi especificado e foi elaborado o diagrama de componentes, definido pela UML, para ilustrar os pontos de comunicação e componentização da solução completa. Nessa etapa foram definidas as tecnologias a serem empregadas em cada módulo, a justificativa das escolhas é apresentada após o detalhamento dos componentes da arquitetura de software. O projeto de implantação foi então realizado, com planejamento da disponibilização do sistema de software com ferramentas de computação de nuvem disponíveis no mercado.

\subsection{Especificação dos Casos de Uso}

Na definição dos casos de uso do projeto foram definidos dois atores que interagem com o sistema de software a ser desenvolvido, identificados como 
usuário e dispositivo. O usuário representa o utilizador humano do sistema a ser desenvolvido, responsável por todas as interações humanas necessárias. 
O usuário se comunica com o sistema por duas interfaces, uma aplicação web, que se comunica diretamente com o sistema, e um aplicativo para smartphone, necessário para configurações iniciais do dispositivo. O dispositivo representa o sistema hardware de controle e monitoramento, também desenvolvido neste projeto. Em relação ao sistema de software, ele é tratado como um ator com suas devidas interações; seu projeto e especificações são discutidos na seção destinado ao projeto de hardware.

Segue a especificação dos casos de uso em si, contendo a identificação de cada caso de uso, sua breve descrição, enumeração dos passos que o definem e 
listagem dos requisitos técnicos relacionados ao caso de uso. Com caráter ilustrativo, o Diagrama de Casos de Uso da UML é apresentado na figura \ref{fig:diagrama_caso_de_usos}.

\subsubsection*{UC - 1: Registro de Dispositivo} 

Descrição: ao obter um novo dispositivo, o usuário deve configurar seu acesso à rede Wi-Fi e registrá-lo, de modo que o sistema reconheça que aquele 
dispositivo pertence ao usuário.
\begin{enumerate}
    \item Usuário acessa aplicativo em seu smartphone
    \item Sistema autentica acesso do usuário
    \item Aplicativo se conecta ao dispositivo
    \item Usuário informa configurações da rede Wi-Fi
    \item Aplicativo envia informações da rede para dispositivo
    \item Dispositivo se conecta na rede e se prepara para receber mensagens do sistema
    \item Aplicativo envia informações do dispositivo para o sistema
    \item Sistema cadastra informações do dispositivo e usuário
\end{enumerate}
Requisito relacionado: SW-F-12

\subsubsection*{UC - 2: Cadastro de Receitas}
Descrição: fluxo de cadastro de receitas.
\begin{enumerate}
    \item Usuário acessa tela de listagem de receitas
    \item Sistema exibe todas as receitas referentes ao usuário
    \item Usuário seleciona opção “Criar Receita” e acessa tela de cadastro de receita
    \item Usuário informa nome, estilo e observações da receita e clica em “Salvar”
    \item Sistema cadastra a receita no banco de dados
\end{enumerate}
Requisito relacionado: SW-F-4

\subsubsection*{UC - 3: Cadastro de Lotes}
Descrição: fluxo de cadastro de lotes, perfis de controle e associação de lote a um dispositivo.
\begin{enumerate}
    \item Usuário acessa tela de listagens de receitas
    \item Sistema exibe todas as receitas referentes ao usuário
    \item Usuário escolhe uma receita e seleciona a opção “Criar Lote”, e acessa a tela de cadastro de lote
    \item Sistema carrega listagem de perfis de controle já cadastrados e dispositivos do usuário
    \item Usuário informa identificação e observações do lote
    \item Usuário escolhe um perfil de controle já existente ou cria um novo perfil, informando uma identificação e cada um dos passos de controle (instante e valor alvo de temperatura)
    \item Usuário seleciona qual dispositivo irá controlar a produção do lote
    \item Usuário clica em “Salvar”
    \item Sistema cadastra lote, perfil de controle (caso novo), associação de lote e perfil de controle, e associação de lote e dispositivo
    \item Sistema envia informações do lote para dispositivo selecionado
\end{enumerate}
Requisitos relacionados: SW-F-5 e SW-F-6.

\subsubsection*{UC - 4: Envio das informações do Lote para Dispositivo}
Descrição: fluxo de envio das informações do lote para o dispositivo associado ao controle daquele lote.
\begin{enumerate}
    \item Dispositivo recebe informações do lote que foi associado por tópico de mensagens
    \item Dispositivo salva informações localmente
    \item Quando pronto, dispositivo inicia rotina de monitoramento e controle
\end{enumerate}    
Requisito relacionado: HW-F-5

\subsubsection*{UC - 5: Envio das informações do Dispositivo para o Sistema}
Descrição: fluxo de envio das informações obtidas pelo monitoramento do processo pelo dispositivo para o sistema.
\begin{enumerate}
    \item Dispositivo envia dados coletados para o sistema
    \item Sistema processa dados e salva informações no banco de dados
\end{enumerate}    
Requisitos relacionados: HW-F-6, SW-F-9

\subsubsection*{UC - 6: Visualização das Informações dos Lotes}
Descrição: fluxo para visualização das informações gerais e de evolução dos lotes correntes e passados
\begin{enumerate}
    \item Usuário acessa tela de listagem dos lotes
    \item Sistema exibe todas os lotes referentes ao usuário
    \item Usuário escolhe um lote e seleciona a opção “Ver Informações”, acessando a tela de informações do lote
    \item Sistema exibe informações gerais sobre o lote, como identificação, receita, observações, status, data de início, data de término, densidades relativas inicial e final/atual, estimativa de teor alcoólico, pH final/atual, temperatura final/atual.
    \item Sistema exibe um gráfico com as variáveis monitoradas em relação ao tempo
    \item Caso usuário clique em “Baixar Dados”, sistema efetua o download dos dados do gráfico em arquivo de texto
\end{enumerate}    
Requisitos relacionados: SW-F-2, SW-F-3, SW-F-8, SW-F-11

\begin{figure}[ht]
    \centering
    \includegraphics[scale=0.45]{figuras/projeto/software/diagrama_casos_de_uso.png}
    \caption{Diagrama de Casos de Uso.}
    \label{fig:diagrama_caso_de_usos}
\end{figure}

\subsection{Modelo de Entidade-Relacionamento}

O Modelo Entidade Relacionamento descreve como as informações são organizada no sistema e como o banco de dados do sistema é estruturado. 
Cada entidade no modelo representa um tipo de informação que é produzida e consumida pelo sistema, na execução dos casos de uso. 

Todas as informações são armazenadas em um banco de dados relacional, dividido em schemas que atuam como partições de dados em diferentes domínios. 
Foram definidos três domínios, traduzidos em schemas no banco de dados, para os dados: user, recipe e control. O schema user contém as informações 
referentes aos usuários e dispositivos do sistema; o domínio recipe contempla as entidades relacionadas às receitas e lotes de produção, assim como 
os dados coletados em cada execução de uma receita; e o schema control, por fim, os perfis de controle que são seguidos pelo dispositivo durante 
seu funcionamento. A separação das entidades em domínios é importante na arquitetura de microsserviços para assegurar que cada serviço tenha 
controle apenas às informações de sua competência. 

A definição das entidades e seus relacionamentos é ilustrada pelo Diagrama Entidade Relacionamento da figura \ref{fig:diagrama_entidade_relacionamento}, com destaque em cor para cada um 
dos schemas determinados.

\begin{figure}[ht]
    \centering
    \includegraphics[scale=0.15]{figuras/projeto/software/banco_de_dados.png}
    \caption{Diagrama de Entidade-Relacionamento.}
    \label{fig:diagrama_entidade_relacionamento}
\end{figure}

\subsection{Arquitetura de Software}

A arquitetura de software do projeto foi estruturada tendo como base o fluxo da informação como especificado nos casos de uso e os conceitos de arquitetura de microsserviços. Dessa forma, foram definidas quatro camadas para organizar os sistemas a serem implementados: Camada de Interface, contemplando as interfaces utilizadas diretamente pelos atores; Camada Intermediária, contendo um API Gateway para isolar os microsserviços das interfaces e um message broker para intermediar a comunicação entre dispositivo e sistema; Camada de Negócio, contendo os microsserviços que exercem as regras de negócio do sistema; e Camada de Persistência, com os serviços de armazenamento de dados. Os componentes de cada camada são descritos em detalhes quanto a suas responsabilidades e detalhes de implementação em sequência. A arquitetura completa é representada visualmente pelo Diagrama de Componentes da figura \ref{fig:diagrama_componentes}.

%% TODO: Diagrama complexo em apêndice

\subsubsection{Camada de Interface}
A Camada de Interface contém os componentes que fornecem interface direta aos atores definidos na modelagem de casos de uso. 

Aplicação Front End
    - Implementação: HTML + CSS + React.js
    - Responsabilidades: disponibilizar interfaces com o usuário para gerenciamento e visualização de suas informações

Aplicativo em SmartPhone
    - Implementação: aplicativo em Java para Android
    - Responsabilidades: envio de configurações iniciais ao dispositivo por bluetooth e registro do dispositivo no sistema

Dispositivo
    - Implementação: linguagem Arduino
    - Responsabilidades: monitoramento e controle do processo de fermentação

\subsubsection{Camada Intermediária}

    A camada intermediária acomoda um serviço de API Gateway, que age como um proxy reverso que roteia as requisições externas ao sistema para quais microsserviços sejam necessários e retorna ao cliente; e um serviço de Message Broker que controla o fluxo de mensagens que são transmitidas entre o dispositivo e o sistema.
    
    API Gateway
        - Implementação: aplicação flask (Python), servido com Gunicorn e Nginx
        - Responsabilidades: Centralizar as requisições ao sistema e traduzi-las em requisições ao microsserviços necessários para o completar a ação solicitada
    
    Message Broker
        - Implementação: Mosquitto (Protocolo MQTT)
        - Responsabilidades: Receber e disponibilizar mensagens em tópicos
    
\subsubsection{Camada de Negócio}
    
\subsubsection{Camada de Persistência}

\begin{figure}[ht]
    \centering
    \includegraphics[scale=0.50, angle=270]{figuras/projeto/software/diagrama_componentes.png}
    \caption{Diagrama de Componentes.}
    \label{fig:diagrama_componentes}
\end{figure}

\subsection{Definição de Tecnologias}

\subsection{Projeto de Implantação}

\subsection{Protótipos das Interfaces}
%% TODO: Apêndices

