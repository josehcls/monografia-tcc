\chapter{Introdução}

\section{Objetivo}

O objetivo geral deste trabalho foi desenvolver um protótipo de um sistema que realize o monitoramento e controle do processo de fermentação de cervejas.
O objetivo específico foi possibilitar às cervejarias de pequeno e médio porte o desenvolvimento da capacidade de testes de novas receitas de cervejas, 
a fim de garantir a reprodutibilidade e qualidade das mesmas por meio do controle do processo de fermentação.

% % // TODO: Escrever Motivação
% \section{Motivação}

\section{Organização do Trabalho}

O presente trabalho é dividido de forma linear, em partes que representam os passos percorridos até a sua conclusão. A primeira parte consiste na contextualização, onde são apontados os pontos principais sobre a cerveja e a sua produção, com um enfoque no processo de fermentação. Após isso, será apresentado o seu projeto, onde são definidas as bases teóricas. Nessa sessão serão enumerados os requisitos funcionais e não funcionais do projeto, que servirão de guia para o resto do projeto de software e hardware. A próxima parte apresentada será a de implementação com a montagem e codificação e readaptação do projeto. Por último, são discutidos os testes de validação sobre o protótipo e feitas as considerações finais sobre o projeto.
