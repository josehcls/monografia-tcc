\chapter{Introdução}

\section{Objetivo}

O objetivo geral deste estudo é desenvolver um protótipo de um sistema que realize o monitoramento e controle do processo de fermentação de cervejas.
O objetivo específico é possibilitar às cervejarias de pequeno e médio porte o desenvolvimento da capacidade de testes de novas receitas de cervejas, 
a fim de garantir a reprodutibilidade e qualidade das mesmas por meio do controle do processo de fermentação.

\section{Resumo}
Na produção de cervejas, a fermentação é um complexo processo bioquímico cuja função primária é converter os açúcares obtidos dos grãos maltados em etanol, tendo grande impacto no sabor, aroma, aparência e textura do produto. 
As principais variáveis da fermentação são a densidade do líquido em fermentação, que indica a evolução do processo, e a temperatura do líquido, que impacta diretamente no funcionamento da levedura de modo que um controle preciso evitar subprodutos indesejados e contaminações que impactam significativamente qualidade final do produto. 
O objetivo geral desse projeto é criar um protótipo de um sistema que realize o monitoramento dessas variáveis no processo e controle de temperatura, de modo a garantir resultados mais precisos e reprodutíveis, voltado para cervejarias de pequeno e médio porte que realizem testes de novas receitas e necessitem de alto grau de controle para escalas pequenas de produção. 
O sistema idealizado é composto por dois principais subsistemas: (i) um físico, agregando o controlador, hardware com sensores e atuadores e software embarcado; e (ii) um digital, que capta, processa e disponibiliza todas as informações adquiridas para o usuário. A execução do projeto será guiada por várias iterações curtas e prototipagem, focando na resolução de algumas milestones a cada iteração, com a finalidade de acelerar a obtenção de resultados.
Deste modo, para a aquisição de conhecimento técnico e de negócio necessários para a execução do projeto será feito um levantamento e estudo de material bibliográfico adequado, entrevistas com mestres-cervejeiros para o qual o produto se destinaria e análise de soluções já existentes no mercado, dentro e fora do país.

% // TODO: Escrever Motivação

\section{Organização do Trabalho}

% // TODO: Escrever Organização do Trabalho

