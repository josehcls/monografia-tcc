\section{Implementação de Software}

A implementação do software seguiu a estratégia de se implementar primeiramente componentes que não possuem dependência para funcionar, seguindo para os componentes que tem como dependência os que foram implementados e assim por diante. De forma geral, seguindo o diagrama de componentes da figura \ref{fig:diagrama_componentes} no sentido da direita para esquerda, foram implementadas em ordem, as camadas de Banco de Dados, de Negócio, Intermediária e de Interface. 


Além disso, foram priorizados serviços essenciais para o funcionamento geral do projeto, i.e., os componentes de Aplicação Mobile e Autenticação, que garantem melhoras na utilização do usuário e segurança da informação, foram considerados menos prioritários que os demais componentes, que atuam desempenham a função primária do projeto, que é controlar e monitorar a temperatura de fermentação de cervejas.


Os links de acesso para os repositórios de código fonte por componente são listados no apêndice \ref{appendice:repositorios}. Eles foram salvos na plataforma Github e serão mantidos públicos para consulta e oportunidade de continuidade do projeto, desde que seguindo os termos de licença especificados na plataforma.


\subsection{Camada de Persistência}


A Camada de Persistência foi rapidamente implementada instanciando-se um banco de dados na nuvem da AWS (Amazon Web Services) pelo serviço RDS e executando o script de DDL (Data Definition Language) que foi gerado a partir do diagrama da figura \ref{fig:diagrama_entidade_relacionamento}. A plataforma utilizada para confecionar o diagrama (\url{https://dbdiagram.io/}) oferece o download do código de geração do banco de dados a partir do diagrama. Foram então realizadas algumas modificações no script gerado, para incluir a criação de índices e estabelecer o relacionamento entre as tabelas.  


O arquivo de extensão \textit{sql} para configuração do banco de dados está disponível no seguinte repositório de códigos da plataforma Github: \url{https://github.com/josehcls/tcc-config}


\subsection{Camada de Negócio}


\subsubsection{Servidores de Registro e Receitas}


Após a configuração do Banco de Dados, foram implementados os componentes responsáveis por transacionar os registros de Usuários, Dispositivos, Receitas, Lotes e Perfis de Controle. 
Os servidores de Registro (\url{https://github.com/josehcls/tcc-register-service}) e Receitas (\url{https://github.com/josehcls/tcc-recipe-service}) foram implementados na linguagem Java e realizam as funções básicas de transação de dados: listagem, criação, atualização e deleção. 


O framework Spring foi utilizado para prover uma interface de programação REST, que serve de ponto de comunicação com o Servidor de Aplicação, ou API Gateway do projeto. Os servidores recebem solicitações de transação por meio dessa interface, realizam validações de consistências de dados e realizam as operações conectando-se diretamente ao Banco de Dados. Para validar as operações, foram criados testes automatizados durante o desenvolvimento, que podem ser executados para verificar a eficácia de cada um dos métodos implementados.


Após a implementação e testes, foi confecionado um arquivo Dockerfile responsável por configurar a geração de um container Docker a partir do pacote em extensão \textit{jar} resultado da compilação de cada um dos componentes.
Esse container é utilizado para simplificar e otimizar a implantação desses componentes, de forma que seja fácil e rápido provisionar uma instância do serviço sempre que necessário.


\subsubsection{Servidor de Processamento}


\subsection{Camada de Aplicaçáo} 
% app para juntar serviços

\subsection{Interface Web}
% front end

\subsection{Implantação}
% docker
