\chapter{Conclusão}

\section{Considerações Finais}

Ao final do presente trabalho, temos em mãos um dispositivo que coleta dados de temperatura, densidade e pH e os envia 
para um servidor online e que aplica um algoritmo de controle sobre a temperatura, e também um sistema de software 
que recebe os dados coletados e os exibe em poucos segundos em um gráfico online. 
Apesar do controle ainda necessitar de melhorias de montagem e isolamento térmico para atingir a potência esperada, e o 
software desenvolvido não ter atendido completamente o processo, como implementação de controle de acesso e o fluxo para 
configuração do dispositivo previsto nos casos de uso; acreditamos que temos um protótipo viável com um caminho evidente de melhorias.

Contudo, lamentamos de não termos dedicado o empenho que gostaríamos na etapa de teste e validação do protótipo. Essa etapa
foi prejudica por atrasos no cronograma, principalmente no momento intermediário do projeto. Com certeza, um controle mais rígido
das tarefas e disciplina com o cronograma teria afetado positivamente o projeto, especialmente nessa etapa.

Por outro lado, as etapas de projeto e prototipação em etapas foram consideravelmente proveitosas como uma aplicação
prática e extensa dos conteúdos que aprendemos durante nossa graduação. Ter atuado nesse trabalho, que se iniciou com
a busca de entendimento sobre um processo, seguindo por levantamento de idéias de melhoria e projeto de uma solução e 
finalizando pela implementação configurou um grande aprendizado de engenharia.

No mais, considerando o objetivo do projeto de interesse pessoal dos autores, esperamos dar continuidade ao trabalho após
a entrega desse documento, e esperamos que esse relatório possa estimular outros alunos interessados e ajudá-los na 
elaboração de seus projetos.

\section{Perspectivas de Continuidade}

Assim como foi mencionado, notamos nesse documento algumas melhorias que podem ser feitas sobre o escopo do trabalho atual,
como melhorar a aplicação do controle de temperatura, refinar a montagem do dispositivo, finalizar a implementação da Interface
Web com controle de acessos e dados segregados por usuário, e refinar as interações entre hardware e software focando nos requisitos
não funcionais levantados. Acreditamos que com poucas semanas adicionais, boa parte disso poderia ser realizado.

Sob a ótica acadêmica, uma continuação interessante ao trabalho seria estudar a viabilidade de estruturar o algoritmo de controle
levando em consideração a densidade do mosto em fermentação. Isso poderia permitir inclusive reduzir o tempo de fermentação em
alguns dias, como discutido brevemente por \citeonline{YeastWhite}, e possui grande valor para cervejarias, mesmo de grande porte 
que poderiam adaptar a estratégia a sua escala. Contudo, para essa abordagem, acreditamos ser necessário um maior compreendimento do 
processo bioquímico e da natureza das leveduras.

Sob a ótica de mercado, aplicando-se um refinamento sobre o protótipo deve ser possível transformá-lo em um produto, assim como
imaginamos no início desse projeto. Os devidos estudo sobre mercado deveriam ser realizados, e seria interessante mais entrevistas 
com produtores do que conseguimos realizar para validar o produto, mas acreditamos sê-lo viável, ainda que para um mercado de nicho.


% \section{Contribuições}
% \textcolor{red}{\{Apresentar as contribuições  trabalho, ressaltando o que foi autoria do grupo\}}
